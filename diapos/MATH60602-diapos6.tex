% Options for packages loaded elsewhere
\PassOptionsToPackage{unicode}{hyperref}
\PassOptionsToPackage{hyphens}{url}
%
\documentclass[
  ignorenonframetext,
]{beamer}
\usepackage{pgfpages}
\setbeamertemplate{caption}[numbered]
\setbeamertemplate{caption label separator}{: }
\setbeamercolor{caption name}{fg=normal text.fg}
\beamertemplatenavigationsymbolsempty
% Prevent slide breaks in the middle of a paragraph
\widowpenalties 1 10000
\raggedbottom
\setbeamertemplate{part page}{
  \centering
  \begin{beamercolorbox}[sep=16pt,center]{part title}
    \usebeamerfont{part title}\insertpart\par
  \end{beamercolorbox}
}
\setbeamertemplate{section page}{
  \centering
  \begin{beamercolorbox}[sep=12pt,center]{part title}
    \usebeamerfont{section title}\insertsection\par
  \end{beamercolorbox}
}
\setbeamertemplate{subsection page}{
  \centering
  \begin{beamercolorbox}[sep=8pt,center]{part title}
    \usebeamerfont{subsection title}\insertsubsection\par
  \end{beamercolorbox}
}
\AtBeginPart{
  \frame{\partpage}
}
\AtBeginSection{
  \ifbibliography
  \else
    \frame{\sectionpage}
  \fi
}
\AtBeginSubsection{
  \frame{\subsectionpage}
}

\usepackage{amsmath,amssymb}
\usepackage{lmodern}
\usepackage{iftex}
\ifPDFTeX
  \usepackage[T1]{fontenc}
  \usepackage[utf8]{inputenc}
  \usepackage{textcomp} % provide euro and other symbols
\else % if luatex or xetex
  \usepackage{unicode-math}
  \defaultfontfeatures{Scale=MatchLowercase}
  \defaultfontfeatures[\rmfamily]{Ligatures=TeX,Scale=1}
  \setmainfont[]{D-DIN}
  \setsansfont[]{Latin Modern Sans}
  \setmathfont[]{Latin Modern Math}
\fi
\usecolortheme{Flip}
\usefonttheme{serif} % use mainfont rather than sansfont for slide text
\useinnertheme{Flip}
\useoutertheme{Flip}
% Use upquote if available, for straight quotes in verbatim environments
\IfFileExists{upquote.sty}{\usepackage{upquote}}{}
\IfFileExists{microtype.sty}{% use microtype if available
  \usepackage[]{microtype}
  \UseMicrotypeSet[protrusion]{basicmath} % disable protrusion for tt fonts
}{}
\makeatletter
\@ifundefined{KOMAClassName}{% if non-KOMA class
  \IfFileExists{parskip.sty}{%
    \usepackage{parskip}
  }{% else
    \setlength{\parindent}{0pt}
    \setlength{\parskip}{6pt plus 2pt minus 1pt}}
}{% if KOMA class
  \KOMAoptions{parskip=half}}
\makeatother
\usepackage{xcolor}
\newif\ifbibliography
\setlength{\emergencystretch}{3em} % prevent overfull lines
\setcounter{secnumdepth}{-\maxdimen} % remove section numbering

\usepackage{color}
\usepackage{fancyvrb}
\newcommand{\VerbBar}{|}
\newcommand{\VERB}{\Verb[commandchars=\\\{\}]}
\DefineVerbatimEnvironment{Highlighting}{Verbatim}{commandchars=\\\{\}}
% Add ',fontsize=\small' for more characters per line
\usepackage{framed}
\definecolor{shadecolor}{RGB}{241,243,245}
\newenvironment{Shaded}{\begin{snugshade}}{\end{snugshade}}
\newcommand{\AlertTok}[1]{\textcolor[rgb]{0.68,0.00,0.00}{#1}}
\newcommand{\AnnotationTok}[1]{\textcolor[rgb]{0.37,0.37,0.37}{#1}}
\newcommand{\AttributeTok}[1]{\textcolor[rgb]{0.40,0.45,0.13}{#1}}
\newcommand{\BaseNTok}[1]{\textcolor[rgb]{0.68,0.00,0.00}{#1}}
\newcommand{\BuiltInTok}[1]{\textcolor[rgb]{0.00,0.23,0.31}{#1}}
\newcommand{\CharTok}[1]{\textcolor[rgb]{0.13,0.47,0.30}{#1}}
\newcommand{\CommentTok}[1]{\textcolor[rgb]{0.37,0.37,0.37}{#1}}
\newcommand{\CommentVarTok}[1]{\textcolor[rgb]{0.37,0.37,0.37}{\textit{#1}}}
\newcommand{\ConstantTok}[1]{\textcolor[rgb]{0.56,0.35,0.01}{#1}}
\newcommand{\ControlFlowTok}[1]{\textcolor[rgb]{0.00,0.23,0.31}{#1}}
\newcommand{\DataTypeTok}[1]{\textcolor[rgb]{0.68,0.00,0.00}{#1}}
\newcommand{\DecValTok}[1]{\textcolor[rgb]{0.68,0.00,0.00}{#1}}
\newcommand{\DocumentationTok}[1]{\textcolor[rgb]{0.37,0.37,0.37}{\textit{#1}}}
\newcommand{\ErrorTok}[1]{\textcolor[rgb]{0.68,0.00,0.00}{#1}}
\newcommand{\ExtensionTok}[1]{\textcolor[rgb]{0.00,0.23,0.31}{#1}}
\newcommand{\FloatTok}[1]{\textcolor[rgb]{0.68,0.00,0.00}{#1}}
\newcommand{\FunctionTok}[1]{\textcolor[rgb]{0.28,0.35,0.67}{#1}}
\newcommand{\ImportTok}[1]{\textcolor[rgb]{0.00,0.46,0.62}{#1}}
\newcommand{\InformationTok}[1]{\textcolor[rgb]{0.37,0.37,0.37}{#1}}
\newcommand{\KeywordTok}[1]{\textcolor[rgb]{0.00,0.23,0.31}{#1}}
\newcommand{\NormalTok}[1]{\textcolor[rgb]{0.00,0.23,0.31}{#1}}
\newcommand{\OperatorTok}[1]{\textcolor[rgb]{0.37,0.37,0.37}{#1}}
\newcommand{\OtherTok}[1]{\textcolor[rgb]{0.00,0.23,0.31}{#1}}
\newcommand{\PreprocessorTok}[1]{\textcolor[rgb]{0.68,0.00,0.00}{#1}}
\newcommand{\RegionMarkerTok}[1]{\textcolor[rgb]{0.00,0.23,0.31}{#1}}
\newcommand{\SpecialCharTok}[1]{\textcolor[rgb]{0.37,0.37,0.37}{#1}}
\newcommand{\SpecialStringTok}[1]{\textcolor[rgb]{0.13,0.47,0.30}{#1}}
\newcommand{\StringTok}[1]{\textcolor[rgb]{0.13,0.47,0.30}{#1}}
\newcommand{\VariableTok}[1]{\textcolor[rgb]{0.07,0.07,0.07}{#1}}
\newcommand{\VerbatimStringTok}[1]{\textcolor[rgb]{0.13,0.47,0.30}{#1}}
\newcommand{\WarningTok}[1]{\textcolor[rgb]{0.37,0.37,0.37}{\textit{#1}}}

\providecommand{\tightlist}{%
  \setlength{\itemsep}{0pt}\setlength{\parskip}{0pt}}\usepackage{longtable,booktabs,array}
\usepackage{calc} % for calculating minipage widths
\usepackage{caption}
% Make caption package work with longtable
\makeatletter
\def\fnum@table{\tablename~\thetable}
\makeatother
\usepackage{graphicx}
\makeatletter
\def\maxwidth{\ifdim\Gin@nat@width>\linewidth\linewidth\else\Gin@nat@width\fi}
\def\maxheight{\ifdim\Gin@nat@height>\textheight\textheight\else\Gin@nat@height\fi}
\makeatother
% Scale images if necessary, so that they will not overflow the page
% margins by default, and it is still possible to overwrite the defaults
% using explicit options in \includegraphics[width, height, ...]{}
\setkeys{Gin}{width=\maxwidth,height=\maxheight,keepaspectratio}
% Set default figure placement to htbp
\makeatletter
\def\fps@figure{htbp}
\makeatother

\usepackage{amsmath}
\usepackage{booktabs}
\usepackage{caption}
\usepackage{longtable}
\usepackage{tabu}
\usepackage{mathtools}
\usepackage{mathrsfs}
\makeatletter
\makeatother
\makeatletter
\makeatother
\makeatletter
\@ifpackageloaded{caption}{}{\usepackage{caption}}
\AtBeginDocument{%
\ifdefined\contentsname
  \renewcommand*\contentsname{Table of contents}
\else
  \newcommand\contentsname{Table of contents}
\fi
\ifdefined\listfigurename
  \renewcommand*\listfigurename{List of Figures}
\else
  \newcommand\listfigurename{List of Figures}
\fi
\ifdefined\listtablename
  \renewcommand*\listtablename{List of Tables}
\else
  \newcommand\listtablename{List of Tables}
\fi
\ifdefined\figurename
  \renewcommand*\figurename{Figure}
\else
  \newcommand\figurename{Figure}
\fi
\ifdefined\tablename
  \renewcommand*\tablename{Table}
\else
  \newcommand\tablename{Table}
\fi
}
\@ifpackageloaded{float}{}{\usepackage{float}}
\floatstyle{ruled}
\@ifundefined{c@chapter}{\newfloat{codelisting}{h}{lop}}{\newfloat{codelisting}{h}{lop}[chapter]}
\floatname{codelisting}{Listing}
\newcommand*\listoflistings{\listof{codelisting}{List of Listings}}
\makeatother
\makeatletter
\@ifpackageloaded{caption}{}{\usepackage{caption}}
\@ifpackageloaded{subcaption}{}{\usepackage{subcaption}}
\makeatother
\makeatletter
\@ifpackageloaded{tcolorbox}{}{\usepackage[many]{tcolorbox}}
\makeatother
\makeatletter
\@ifundefined{shadecolor}{\definecolor{shadecolor}{rgb}{.97, .97, .97}}
\makeatother
\makeatletter
\makeatother
\ifLuaTeX
  \usepackage{selnolig}  % disable illegal ligatures
\fi
\IfFileExists{bookmark.sty}{\usepackage{bookmark}}{\usepackage{hyperref}}
\IfFileExists{xurl.sty}{\usepackage{xurl}}{} % add URL line breaks if available
\urlstyle{same} % disable monospaced font for URLs
\hypersetup{
  pdftitle={Régression logistique},
  pdfauthor={Léo Belzile},
  hidelinks,
  pdfcreator={LaTeX via pandoc}}

\title{Régression logistique}
\subtitle{Analyse multidimensionnelle appliquée}
\author{Léo Belzile}
\date{automne 2022}
\institute{HEC Montréal}

\begin{document}
\frame{\titlepage}
\ifdefined\Shaded\renewenvironment{Shaded}{\begin{tcolorbox}[boxrule=0pt, sharp corners, enhanced, interior hidden, breakable, frame hidden, borderline west={3pt}{0pt}{shadecolor}]}{\end{tcolorbox}}\fi

\begin{frame}{\emph{Professional Rodeo Cowboys Association}}
\protect\hypertarget{cowboy}{}
L'exemple suivant est inspiré de l'article

\begin{quote}
Daneshvary, R. et Schwer, R. K. (2000) The Association Endorsement and
Consumers' Intention to Purchase. \emph{Journal of Consumer Marketing}
\textbf{17}, 203-213.
\end{quote}

\textbf{Objectif}: Les auteurs cherchent à voir si le fait qu'un produit
soit recommandé par le \emph{Professional Rodeo Cowboys Association}
(PRCA) a un effet sur les intentions d'achats.
\end{frame}

\begin{frame}[fragile]{Données du PRCA}
\protect\hypertarget{donnuxe9es-du-prca}{}
On dispose de 500 observations sur les variables suivantes dans la base
de données \texttt{logit1}: \footnotesize 

\begin{itemize}
\tightlist
\item
  \(Y\): seriez-vous intéressé à acheter un produit recommandé par le
  PRCA

  \begin{itemize}
  \tightlist
  \item
    \(\texttt{0}\): non
  \item
    \(\texttt{1}\): oui
  \end{itemize}
\item
  \(\mathrm{X}_1\): quel genre d'emploi occupez-vous?

  \begin{itemize}
  \tightlist
  \item
    \(\texttt{1}\): à la maison
  \item
    \(\texttt{2}\): employé
  \item
    \(\texttt{3}\): ventes/services
  \item
    \(\texttt{4}\): professionnel
  \item
    \(\texttt{5}\): agriculture/ferme
  \end{itemize}
\item
  \(\mathrm{X}_2\): revenu familial annuel

  \begin{itemize}
  \tightlist
  \item
    \(\texttt{1}\): moins de 25 000
  \item
    \(\texttt{2}\): 25 000 à 39 999
  \item
    \(\texttt{3}\): 40 000 à 59 999
  \item
    \(\texttt{4}\): 60 000 à 79 999
  \item
    \(\texttt{5}\): 80 000 et plus
  \end{itemize}
\end{itemize}

\normalsize
\end{frame}

\begin{frame}{Données du PRCA}
\protect\hypertarget{donnuxe9es-du-prca-1}{}
\footnotesize

\begin{itemize}
\tightlist
\item
  \(\mathrm{X}_3\): sexe

  \begin{itemize}
  \tightlist
  \item
    \(\texttt{0}\): homme
  \item
    \(\texttt{1}\): femme
  \end{itemize}
\item
  \(\mathrm{X}_4\): avez-vous déjà fréquenté une université?

  \begin{itemize}
  \tightlist
  \item
    \(\texttt{0}\): non
  \item
    \(\texttt{1}\): oui
  \end{itemize}
\item
  \(\mathrm{X}_5\): âge (en années)
\item
  \(\mathrm{X}_6\): combien de fois avez-vous assisté à un rodéo au
  cours de la dernière année?

  \begin{itemize}
  \tightlist
  \item
    \(\texttt{1}\): 10 fois ou plus
  \item
    \(\texttt{2}\): entre six et neuf fois
  \item
    \(\texttt{3}\): cinq fois ou moins
  \end{itemize}
\end{itemize}

\normalsize
\end{frame}

\begin{frame}{Régression logistique}
\protect\hypertarget{ruxe9gression-logistique}{}
Expliquer le comportement de la \textbf{moyenne} d'une variable binaire
\(Y\in\{0,1\}\) en utilisant un modèle de régression avec \(p\)
variables explicatives \(\mathrm{X}_1, \ldots, \mathrm{X}_p\).

\[\underset{\text{moyenne théorique}}{\mathsf{E}(Y=1 \mid \mathbf{X})} = \underset{\text{probabilité de succès}}{\Pr(Y=1 \mid \mathbf{X})}=p\]
\end{frame}

\begin{frame}{Objectif de la régression}
\protect\hypertarget{objectif-de-la-ruxe9gression}{}
\begin{enumerate}
[1)]
\tightlist
\item
  \textbf{Inférence} : comprendre comment et dans quelles mesures les
  variables \(\mathbf{X}\) influencent la probabilité que \(Y=1\).
\item
  \textbf{Prédiction} : développer un modèle pour prévoir des valeurs de
  \(Y\) ou la probabilité de succès à partir des \(\mathbf{X}\).
\end{enumerate}
\end{frame}

\begin{frame}{Exemples}
\protect\hypertarget{exemples}{}
\begin{itemize}
\tightlist
\item
  Est-ce qu'un client potentiel va répondre favorablement à une offre
  promotionnelle?
\item
  Est-ce qu'un client est satisfait du service après-vente?
\item
  Est-ce qu'un client va faire faillite ou non au cours des trois
  prochaines années.
\end{itemize}
\end{frame}

\begin{frame}{Modéliser une probabilité avec une régression linéaire?}
\protect\hypertarget{moduxe9liser-une-probabilituxe9-avec-une-ruxe9gression-linuxe9aire}{}
\begin{itemize}
\tightlist
\item
  Aucune contrainte sur le prédicteur linéaire
  \(\beta_0 + \beta_1 \mathrm{X}_1 + \cdots + \beta_p \mathrm{X}_p\)

  \begin{itemize}
  \tightlist
  \item
    retourne des probabilités négatives ou supérieures à 1!
  \end{itemize}
\item
  les données binaires ne respectent pas le postulat d'égalité des
  variances

  \begin{itemize}
  \tightlist
  \item
    invalide résultat des tests d'hypothèse sur coefficients.
  \end{itemize}
\end{itemize}
\end{frame}

\begin{frame}{Illustration: linéaire vs logistique}
\protect\hypertarget{illustration-linuxe9aire-vs-logistique}{}
\begin{figure}

{\centering \includegraphics[width=0.8\textwidth,height=\textheight]{MATH60602-diapos6_files/figure-beamer/fig-demandecredit-1.pdf}

}

\caption{\label{fig-demandecredit}Données de la réserve de Boston sur
l'approbation de prêts hypothécaires (1990); données tirées de Stock et
Watson (2007).}

\end{figure}
\end{frame}

\begin{frame}{Fonction de liaison}
\protect\hypertarget{fonction-de-liaison}{}
Idée: appliquer une transformation au \textbf{prédicteur linéaire}
\[\eta = \beta_0 + \beta_1 \mathrm{X}_1 + \cdots + \beta_p \mathrm{X}_p\]
pour que la prédiction soit entre zéro et un.

On considère \begin{align*}
 p &= \textrm{expit}(\eta) = \frac{\exp(\eta)}{1+\exp(\eta)}
= \frac{1}{1+\exp(-\eta)}.
\end{align*}
\end{frame}

\begin{frame}{Courbe sigmoïde}
\protect\hypertarget{courbe-sigmouxefde}{}
\begin{figure}

{\centering \includegraphics[width=0.9\textwidth,height=\textheight]{MATH60602-diapos6_files/figure-beamer/fig-logitplot-1.pdf}

}

\caption{\label{fig-logitplot}Valeurs ajustées du modèle de régression
logistique en fonction du prédicteur linéaire \(\eta\).}

\end{figure}
\end{frame}

\begin{frame}{Cote}
\protect\hypertarget{cote}{}
La cote donne le ratio de la probabilité de succès (\(Y=1\)) sur la
probabilité d'échec (\(Y=0\)).

\begin{align*}
 \mathsf{cote}(p) = \frac{p}{1-p} = \frac{\Pr(Y=1 \mid \mathbf{X})}{\Pr(Y=0 \mid \mathbf{X})}.
\end{align*}
\end{frame}

\begin{frame}{Cotes et probabilités}
\protect\hypertarget{cotes-et-probabilituxe9s}{}
\hypertarget{tbl-cotes}{}
\begin{longtable}[]{@{}
  >{\raggedright\arraybackslash}p{(\columnwidth - 18\tabcolsep) * \real{0.0465}}
  >{\centering\arraybackslash}p{(\columnwidth - 18\tabcolsep) * \real{0.1318}}
  >{\centering\arraybackslash}p{(\columnwidth - 18\tabcolsep) * \real{0.1318}}
  >{\centering\arraybackslash}p{(\columnwidth - 18\tabcolsep) * \real{0.1318}}
  >{\centering\arraybackslash}p{(\columnwidth - 18\tabcolsep) * \real{0.1318}}
  >{\centering\arraybackslash}p{(\columnwidth - 18\tabcolsep) * \real{0.0543}}
  >{\centering\arraybackslash}p{(\columnwidth - 18\tabcolsep) * \real{0.1318}}
  >{\centering\arraybackslash}p{(\columnwidth - 18\tabcolsep) * \real{0.1318}}
  >{\centering\arraybackslash}p{(\columnwidth - 18\tabcolsep) * \real{0.0543}}
  >{\centering\arraybackslash}p{(\columnwidth - 18\tabcolsep) * \real{0.0543}}@{}}
\caption{\label{tbl-cotes}Cote et probabilité de succès}\tabularnewline
\toprule()
\begin{minipage}[b]{\linewidth}\raggedright
(p)
\end{minipage} & \begin{minipage}[b]{\linewidth}\centering
0.1
\end{minipage} & \begin{minipage}[b]{\linewidth}\centering
0.2
\end{minipage} & \begin{minipage}[b]{\linewidth}\centering
0.3
\end{minipage} & \begin{minipage}[b]{\linewidth}\centering
0.4
\end{minipage} & \begin{minipage}[b]{\linewidth}\centering
0.5
\end{minipage} & \begin{minipage}[b]{\linewidth}\centering
0.6
\end{minipage} & \begin{minipage}[b]{\linewidth}\centering
0.7
\end{minipage} & \begin{minipage}[b]{\linewidth}\centering
0.8
\end{minipage} & \begin{minipage}[b]{\linewidth}\centering
0.9
\end{minipage} \\
\midrule()
\endfirsthead
\toprule()
\begin{minipage}[b]{\linewidth}\raggedright
(p)
\end{minipage} & \begin{minipage}[b]{\linewidth}\centering
0.1
\end{minipage} & \begin{minipage}[b]{\linewidth}\centering
0.2
\end{minipage} & \begin{minipage}[b]{\linewidth}\centering
0.3
\end{minipage} & \begin{minipage}[b]{\linewidth}\centering
0.4
\end{minipage} & \begin{minipage}[b]{\linewidth}\centering
0.5
\end{minipage} & \begin{minipage}[b]{\linewidth}\centering
0.6
\end{minipage} & \begin{minipage}[b]{\linewidth}\centering
0.7
\end{minipage} & \begin{minipage}[b]{\linewidth}\centering
0.8
\end{minipage} & \begin{minipage}[b]{\linewidth}\centering
0.9
\end{minipage} \\
\midrule()
\endhead
cote & (\frac{1}{9}) & (\frac{1}{4}) & (\frac{3}{7}) & (\frac{2}{3}) &
(1) & (\frac{3}{2}) & (\frac{7}{3}) & (4) & (9) \\
\bottomrule()
\end{longtable}
\end{frame}

\begin{frame}[fragile]{Ajustement du modèle}
\protect\hypertarget{ajustement-du-moduxe8le}{}
La fonction \texttt{glm} dans \textbf{R} ajuste un modèle linéaire
généralisé (par défaut, Gaussien pour régression linéaire).

\begin{itemize}
\tightlist
\item
  L'argument \texttt{family=binomial(link="logit")} permet de spécifier
  que l'on ajuste un modèle logistique.
\end{itemize}

\begin{Shaded}
\begin{Highlighting}[numbers=left,,]
\FunctionTok{data}\NormalTok{(logit1, }\AttributeTok{package =} \StringTok{"hecmulti"}\NormalTok{)}
\CommentTok{\# Ajustement du modèle avec une}
\CommentTok{\#  seule variable explicative}
\NormalTok{modele1 }\OtherTok{\textless{}{-}} \FunctionTok{glm}\NormalTok{(}\AttributeTok{formula =}\NormalTok{ y }\SpecialCharTok{\textasciitilde{}}\NormalTok{ x5,}
            \AttributeTok{family =} \FunctionTok{binomial}\NormalTok{(}\AttributeTok{link =} \StringTok{"logit"}\NormalTok{),}
            \AttributeTok{data =}\NormalTok{ logit1)}
\end{Highlighting}
\end{Shaded}
\end{frame}

\begin{frame}[fragile]{Sortie}
\protect\hypertarget{sortie}{}
Tableau résumé avec les coefficients (\texttt{summary})

\begin{Shaded}
\begin{Highlighting}[numbers=left,,]
\FunctionTok{summary}\NormalTok{(modele1)}
\end{Highlighting}
\end{Shaded}

Cote pour une augmentation d'une unité des variables explicatives
\(\exp(\widehat{\beta})\)

\begin{Shaded}
\begin{Highlighting}[numbers=left,,]
\NormalTok{cote }\OtherTok{\textless{}{-}} \FunctionTok{exp}\NormalTok{(}\FunctionTok{coef}\NormalTok{(modele1))}
\NormalTok{confrcote }\OtherTok{\textless{}{-}} \FunctionTok{exp}\NormalTok{(}\FunctionTok{confint}\NormalTok{(modele1))}
\end{Highlighting}
\end{Shaded}
\end{frame}

\begin{frame}{Interprétation}
\protect\hypertarget{interpruxe9tation}{}
Par défaut, pour des variables \(0/1\), le modèle décrit la probabilité
de succès.

\begin{figure}

{\centering \includegraphics[width=0.7\textwidth,height=\textheight]{MATH60602-diapos6_files/figure-beamer/logitplot2-1.pdf}

}

\end{figure}

Si le coefficient \(\beta_j\) de la variable \(\mathrm{X}_j\) est
positif, alors plus la variable augmente, plus \(\Pr(Y=1)\) augmente.
\end{frame}

\begin{frame}{Example avec données du PRCA}
\protect\hypertarget{example-avec-donnuxe9es-du-prca}{}
Le modèle ajusté en termes de cote est \begin{align*}
 \frac{\Pr(Y=1 \mid \mathrm{X}_5=x_5)}{\Pr(Y=0 \mid \mathrm{X}_5=x_5)} = \exp(-3.05)\exp(0.0749x_5).
\end{align*}

\small

\begin{itemize}
\tightlist
\item
  Lorsque \(\mathrm{X}_5\) augmente d'une année, la cote est multipliée
  par \(\exp(0.0749) = 1.078\) peut importe la valeur de \(x_5\).
\item
  Pour deux personnes dont la différence d'âge

  \begin{itemize}
  \tightlist
  \item
    est d'un an, la cote de la personne plus âgée est 7.8\% plus élevée
  \item
    est de 10 ans, la cote de la personne plus âgée est 112\% plus
    élevées (cote est multiplié par \(1.078^{10} = 2.12\))
  \end{itemize}
\end{itemize}

\normalsize
\end{frame}

\begin{frame}{Vraisemblance et estimation du modèle}
\protect\hypertarget{vraisemblance-et-estimation-du-moduxe8le}{}
Pour un modèle probabiliste donné, on peut calculer la « probabilité »
d'avoir obtenu les données de l'échantillon.

Si on traite cette « probabilité » comme une fonction des paramètres, on
l'appelle \textbf{vraisemblance}.

\textbf{Maximum de vraisemblance}: valeurs des paramètres qui maximisent
la fonction de vraisemblance.

\begin{itemize}
\tightlist
\item
  on cherche les valeurs des paramètres qui rendent les données les plus
  plausibles
\end{itemize}
\end{frame}

\begin{frame}{Vraisemblance d'une observation}
\protect\hypertarget{vraisemblance-dune-observation}{}
La vraisemblance d'une observation \(Y_i \in \{0,1\}\) (loi
Bernoulli/binomiale) est

\begin{align*}
L(\boldsymbol{\beta}; y_i) = p_i^{y_i}(1-p_i)^{1-y_i} = \begin{cases} 
p_i & y_i = 1 (\text{succès})\\
1-p_i & y_i = 0 (\text{échec}) 
\end{cases}
\end{align*} et où
\[p_i = \mathrm{expit}(\eta_i) = \frac{\mathrm{exp}(\beta_0 + \beta_1 \mathrm{X}_{i1} + \cdots + \beta_p\mathrm{X}_{ip})}{1+\mathrm{exp}(\beta_0 + \beta_1 \mathrm{X}_{i1} + \cdots + \beta_p\mathrm{X}_{ip})}.\]
\end{frame}

\begin{frame}{Log vraisemblance}
\protect\hypertarget{log-vraisemblance}{}
Pour des questions de stabilité numérique, on maximise le logarithme
naturel \(\ell(\boldsymbol{\beta}) = \ln L(\boldsymbol{\beta})\)
(transformation monotone croissante), qui après simplification s'écrit

\begin{align*}
 \ell(\boldsymbol{\beta}) &= \sum_{i=1}^n Y_i ( \beta_0 + \beta_1 \mathrm{X}_{i1} + \cdots + \beta_p \mathrm{X}_{ip}) \\&- \sum_{i=1}^n \ln\left\{1+\exp(\beta_0 + \cdots + \beta_p\mathrm{X}_{ip})\right\}
\end{align*}

Pas de solution explicite pour
\(\widehat{\beta}_0, \ldots, \widehat{\beta}_p\) dans le cas de la
régression logistique.
\end{frame}

\begin{frame}{Prédiction des probabilités de succès}
\protect\hypertarget{pruxe9diction-des-probabilituxe9s-de-succuxe8s}{}
Des estimés des paramètres \(\widehat{\boldsymbol{\beta}}\), découle une
estimation de \(\Pr(Y=1)\) pour les valeurs
\(\mathrm{X}_1=x_1, \ldots, \mathrm{X}_p=x_p\) d'un individu donné,
\begin{align*}
 \widehat{p} = \textrm{expit}(\widehat{\beta}_0 + \cdots + \widehat{\beta}_p\mathrm{X}_p).
\end{align*}
\end{frame}

\begin{frame}[fragile]{Test du rapport de vraisemblance}
\protect\hypertarget{test-du-rapport-de-vraisemblance}{}
\begin{columns}[T]
\begin{column}{0.6\textwidth}
Pour les modèles ajustés par maximum de vraisemblance.

Comparaison de modèles \textbf{emboîtés}

\begin{itemize}
\tightlist
\item
  Modèle complet (sous l'alternative) avec tous les paramètres
\item
  Modèle restreint (sous l'hypothèse nulle) sur lequel on impose
  \(k\leq p\) restrictions (typiquement
  \(\beta_j = 0, j \in \{1, \ldots, p\}\)).
\end{itemize}
\end{column}

\begin{column}{0.4\textwidth}
\includegraphics[width=1\textwidth,height=\textheight]{figures/poupeesrusses.jpg}
\end{column}
\end{columns}
\end{frame}

\begin{frame}[fragile]{Exemple}
\protect\hypertarget{exemple}{}
Comparons un modèle avec et sans \(X_6\).

Variable catégorielle à trois niveaux (deux coefficients associés à
\(\mathrm{I}(\mathrm{X}_{6}=2)\) et \(\mathrm{I}(\mathrm{X}_{6}=3)\).

\begin{Shaded}
\begin{Highlighting}[numbers=left,,]
\NormalTok{modele2 }\OtherTok{\textless{}{-}}  \FunctionTok{glm}\NormalTok{(y }\SpecialCharTok{\textasciitilde{}}\NormalTok{ x1 }\SpecialCharTok{+}\NormalTok{ x2 }\SpecialCharTok{+}\NormalTok{ x3 }\SpecialCharTok{+}\NormalTok{ x4 }\SpecialCharTok{+}\NormalTok{ x5 }\SpecialCharTok{+}\NormalTok{ x6,}
                 \AttributeTok{data =}\NormalTok{ hecmulti}\SpecialCharTok{::}\NormalTok{logit1,}
                 \AttributeTok{family =} \FunctionTok{binomial}\NormalTok{(}\AttributeTok{link =} \StringTok{"logit"}\NormalTok{))}
\NormalTok{modele3 }\OtherTok{\textless{}{-}}  \FunctionTok{glm}\NormalTok{(y }\SpecialCharTok{\textasciitilde{}}\NormalTok{ x1 }\SpecialCharTok{+}\NormalTok{ x2 }\SpecialCharTok{+}\NormalTok{ x3 }\SpecialCharTok{+}\NormalTok{ x4 }\SpecialCharTok{+}\NormalTok{ x5,}
                 \AttributeTok{data =}\NormalTok{ hecmulti}\SpecialCharTok{::}\NormalTok{logit1,}
                 \AttributeTok{family =} \FunctionTok{binomial}\NormalTok{(}\AttributeTok{link =} \StringTok{"logit"}\NormalTok{)) }
\end{Highlighting}
\end{Shaded}
\end{frame}

\begin{frame}{Rapport de vraisemblance}
\protect\hypertarget{rapport-de-vraisemblance}{}
Le test est basé sur la statistique \begin{align*}
 D = -2\{\ell(\widehat{\boldsymbol{\beta}}_0)-\ell(\widehat{\boldsymbol{\beta}})\}.
\end{align*}

Cette différence \(D\), lorsque l'hypothèse \(\mathscr{H}_0\) est vraie,
suit approximativement une loi khi-deux \(\chi^2_k\).
\end{frame}

\begin{frame}[fragile]{Exemple de test}
\protect\hypertarget{exemple-de-test}{}
\footnotesize

\begin{Shaded}
\begin{Highlighting}[numbers=left,,]
\CommentTok{\# modèle 2 (alternative), modèle 3 (nulle)}
\FunctionTok{anova}\NormalTok{(modele3, modele2, }\AttributeTok{test =} \StringTok{"LR"}\NormalTok{)}
\end{Highlighting}
\end{Shaded}

\begin{verbatim}
Analysis of Deviance Table

Model 1: y ~ x1 + x2 + x3 + x4 + x5
Model 2: y ~ x1 + x2 + x3 + x4 + x5 + x6
  Resid. Df Resid. Dev Df Deviance  Pr(>Chi)    
1       488     566.45                          
2       486     516.20  2   50.251 1.225e-11 ***
---
Signif. codes:  0 '***' 0.001 '**' 0.01 '*' 0.05 '.' 0.1 ' ' 1
\end{verbatim}

\begin{Shaded}
\begin{Highlighting}[numbers=left,,]
\DocumentationTok{\#\# Deviance = {-}2*log vraisemblance}
\NormalTok{rvrais }\OtherTok{\textless{}{-}}\NormalTok{ modele3}\SpecialCharTok{$}\NormalTok{deviance }\SpecialCharTok{{-}}\NormalTok{ modele2}\SpecialCharTok{$}\NormalTok{deviance}
\FunctionTok{pchisq}\NormalTok{(rvrais, }\AttributeTok{df =} \DecValTok{2}\NormalTok{, }\AttributeTok{lower.tail =} \ConstantTok{FALSE}\NormalTok{) }\CommentTok{\# valeur{-}p}
\end{Highlighting}
\end{Shaded}

\begin{verbatim}
[1] 1.225046e-11
\end{verbatim}

\normalsize
\end{frame}

\begin{frame}[fragile]{Tester la significativité des variables}
\protect\hypertarget{tester-la-significativituxe9-des-variables}{}
Si un paramètre n'est pas significativement différent de 0, cela veut
dire qu'il n'y a pas de lien significatif entre la variable et la
réponse \emph{une fois que les autres variables} sont dans le modèle.

\footnotesize

\begin{Shaded}
\begin{Highlighting}[numbers=left,,]
\NormalTok{car}\SpecialCharTok{::}\FunctionTok{Anova}\NormalTok{(modele2, }\AttributeTok{type =} \StringTok{"3"}\NormalTok{)}
\end{Highlighting}
\end{Shaded}

\begin{verbatim}
Analysis of Deviance Table (Type III tests)

Response: y
   LR Chisq Df Pr(>Chisq)    
x1    4.291  4     0.3681    
x2   32.912  4  1.245e-06 ***
x3   29.878  1  4.601e-08 ***
x4   42.957  1  5.597e-11 ***
x5   36.731  1  1.356e-09 ***
x6   50.251  2  1.225e-11 ***
---
Signif. codes:  0 '***' 0.001 '**' 0.01 '*' 0.05 '.' 0.1 ' ' 1
\end{verbatim}

\normalsize
\end{frame}

\begin{frame}[fragile]{Intervalles de confiance pour coefficients}
\protect\hypertarget{intervalles-de-confiance-pour-coefficients}{}
On peut fixer la valeur de \(\beta_j\) et maximiser la vraisemblance.

La courbe de vraisemblance profilée résultante permet de déterminer
l'intervalle de confiance pour le paramètre.

\begin{verbatim}
                  2.5 %      97.5 %
(Intercept) -4.61883066 -0.99874984
x12         -1.72420215  0.05426569
x13         -1.54901702  0.19327640
x14         -1.57729645  0.22211588
x15         -1.31245964  0.58527619
x22         -0.97175857  0.59651164
x23         -1.40224186  0.27289938
x24         -3.55887026 -1.36892325
x25         -2.52741920 -0.17813092
x3           0.85055238  1.87143210
x4           1.26006609  2.43001850
x5           0.07265668  0.14832237
x62         -2.05089230 -0.71972379
x63         -3.17313491 -1.69675600
\end{verbatim}

\begin{verbatim}
                  2.5 %     97.5 %
(Intercept) 0.009864324  0.3683396
x12         0.178315264  1.0557651
x13         0.212456712  1.2132181
x14         0.206532716  1.2487161
x15         0.269157210  1.7954868
x22         0.378416980  1.8157737
x23         0.246044748  1.3137680
x24         0.028470971  0.2543807
x25         0.079864870  0.8368329
x3          2.340939573  6.4975949
x4          3.525654492 11.3590922
x5          1.075361280  1.1598867
x62         0.128620085  0.4868867
x63         0.041872127  0.1832771
\end{verbatim}

Les intervalles de confiance de vraisemblance sont invariants aux
reparamétrisation.
\end{frame}

\begin{frame}{Intervalles de confiance}
\protect\hypertarget{intervalles-de-confiance}{}
\begin{figure}

{\centering \includegraphics[width=0.8\textwidth,height=\textheight]{MATH60602-diapos6_files/figure-beamer/fig-confint-modele2-logist-1.pdf}

}

\caption{\label{fig-confint-modele2-logist}Intervalles de confiance
profilés de niveau 95\% pour les coefficients du modèle logistique
(échelle exponentielle).}

\end{figure}
\end{frame}

\begin{frame}{Tests et intervalles de confiances}
\protect\hypertarget{tests-et-intervalles-de-confiances}{}
Comme \(\exp(\cdot)\) est une transformation monotone croissante,
\[\beta>0 \quad \iff \quad \exp(\beta)>1.\]

Si la valeur postulée, par exemple \(\mathscr{H}_0: \beta_j=0\) ou
\(\exp(\beta_j)=1\), est dans l'intervalle de confiance de niveau
\(1-\alpha\), on ne rejette pas l'hypothèse nulle.
\end{frame}

\begin{frame}{Coefficients pour données complètes}
\protect\hypertarget{coefficients-pour-donnuxe9es-compluxe8tes}{}
\footnotesize

\hypertarget{tbl-logit1-complet}{}
\setlength{\LTpost}{0mm}
\begin{longtable}{lccc}
\caption{\label{tbl-logit1-complet}Modèle logistique avec toutes les variables catégorielles. }\tabularnewline

\toprule
variables & cote\textsuperscript{1} & IC 95\%\textsuperscript{1} & valeur-p \\ 
\midrule
x1 &  &  & 0.4 \\ 
1 & — & — &  \\ 
2 & 0.44 & 0.18, 1.06 &  \\ 
3 & 0.51 & 0.21, 1.21 &  \\ 
4 & 0.51 & 0.21, 1.25 &  \\ 
5 & 0.70 & 0.27, 1.80 &  \\ 
x2 &  &  & <0.001 \\ 
1 & — & — &  \\ 
2 & 0.83 & 0.38, 1.82 &  \\ 
3 & 0.57 & 0.25, 1.31 &  \\ 
4 & 0.09 & 0.03, 0.25 &  \\ 
5 & 0.26 & 0.08, 0.84 &  \\ 
x3 &  &  & <0.001 \\ 
0 & — & — &  \\ 
1 & 3.85 & 2.34, 6.50 &  \\ 
\bottomrule
\end{longtable}
\begin{minipage}{\linewidth}
\textsuperscript{1}cote = rapport de cote, IC = intervalle de confiance\\
\end{minipage}

\normalsize
\end{frame}

\begin{frame}{Coefficients pour données complètes}
\protect\hypertarget{coefficients-pour-donnuxe9es-compluxe8tes-1}{}
\footnotesize

\hypertarget{tbl-logit1-complet2}{}
\setlength{\LTpost}{0mm}
\begin{longtable}{lccc}
\caption{\label{tbl-logit1-complet2}Modèle logistique avec toutes les variables catégorielles. }\tabularnewline

\toprule
variables & cote\textsuperscript{1} & IC 95\%\textsuperscript{1} & valeur-p \\ 
\midrule
x4 &  &  & <0.001 \\ 
0 & — & — &  \\ 
1 & 6.24 & 3.53, 11.4 &  \\ 
x5 & 1.12 & 1.08, 1.16 & <0.001 \\ 
x6 &  &  & <0.001 \\ 
1 & — & — &  \\ 
2 & 0.25 & 0.13, 0.49 &  \\ 
3 & 0.09 & 0.04, 0.18 &  \\ 
\bottomrule
\end{longtable}
\begin{minipage}{\linewidth}
\textsuperscript{1}cote = rapport de cote, IC = intervalle de confiance\\
\end{minipage}
\end{frame}

\begin{frame}{Multicolinéarité}
\protect\hypertarget{multicolinuxe9arituxe9}{}
Il est difficile de départager l'effet individuel d'une variable
explicative lorsqu'elle est fortement corrélée avec d'autres.

La multicollinéarité ne dépend pas de la variable réponse \(Y\), mais de
la matrice \(\mathbf{X}\) du modèle.
\end{frame}

\begin{frame}[fragile]{Multicolinéarité pour PRCA}
\protect\hypertarget{multicolinuxe9arituxe9-pour-prca}{}
Mêmes diagnostics qu'en régression linéaire: considérer les facteurs
d'inflation de la variance (\texttt{car::vif}).

\begin{Shaded}
\begin{Highlighting}[numbers=left,,]
\NormalTok{car}\SpecialCharTok{::}\FunctionTok{vif}\NormalTok{(modele2)}
\end{Highlighting}
\end{Shaded}

\begin{verbatim}
       GVIF Df GVIF^(1/(2*Df))
x1 1.698464  4        1.068457
x2 1.852841  4        1.080139
x3 1.450100  1        1.204201
x4 1.491202  1        1.221148
x5 1.219334  1        1.104234
x6 1.179133  2        1.042055
\end{verbatim}

Pas d'inquiétude ici, coefficients faibles (inférieurs à 5)
\end{frame}

\begin{frame}{Dichotomiser des variables continues}
\protect\hypertarget{dichotomiser-des-variables-continues}{}
Si \(Y\) est continue et qu'on cherche à estimer
\(\Pr(Y> c \mid \mathbf{X})\) pour une valeur \(c\) donnée, il n'est
\textbf{pas} recommandé de dichotomiser \(Y\) via

\begin{align*}
Y^{*} = \begin{cases}
1, & Y > c; \\
0, & Y \leq c.
\end{cases}
\end{align*}

et d'ajuster une régression logistique.

Pourquoi? \textbf{On perd de l'information}.
\end{frame}

\begin{frame}[fragile]{Probabilité de dépassement}
\protect\hypertarget{probabilituxe9-de-duxe9passement}{}
On peut estimer plutôt une régression linéaire et prendre
\[\Pr(Y > c \mid \mathbf{X}) = \Phi\left(\frac{\widehat{\mu}-c}{\widehat{\sigma}}\right),\]

où

\begin{itemize}
\tightlist
\item
  \(\widehat{\mu}=\widehat{\beta}_0 + \cdots + \beta_p\mathrm{X}_p\) est
  la moyenne prédite pour le profil donné,
\item
  \(\widehat{\sigma}\) est l'estimation de l'écart-type
\item
  \(\Phi(\cdot)\) est la fonction de répartition d'une loi normale
  standard (\texttt{pnorm})
\end{itemize}
\end{frame}

\begin{frame}{Récapitulatif}
\protect\hypertarget{ruxe9capitulatif}{}
\begin{itemize}
\tightlist
\item
  Une régression logistique sert à modéliser la moyenne de
  \textbf{variables catégorielles}, typiquement binaires.
\item
  C'est un cas particulier d'un modèle de régression linéaire
  généralisée (GLM)
\end{itemize}
\end{frame}

\begin{frame}{Récapitulatif}
\protect\hypertarget{ruxe9capitulatif-1}{}
Le modèle est interprétable à l'échelle de la cote

\begin{itemize}
\tightlist
\item
  La cote donne le rapport probabilité de réussite (1) sur probabilité
  d'échec (0)
\item
  Interprétation en terme de

  \begin{itemize}
  \tightlist
  \item
    pourcentage d'augmentation si \(\exp(\widehat{\eta}) > 1\), avec
    \(\exp(\widehat{\eta})-1\).
  \item
    pourcentage de diminution si \(\exp(\widehat{\eta}) < 1\), avec
    \(1-\exp(\widehat{\eta})\)
  \end{itemize}
\end{itemize}
\end{frame}

\begin{frame}{Récapitulatif}
\protect\hypertarget{ruxe9capitulatif-2}{}
\begin{itemize}
\tightlist
\item
  Estimation par maximum de vraisemblance
\item
  Tests d'hypothèse comparent modèles emboîtés

  \begin{itemize}
  \tightlist
  \item
    loi nulle asymptotique \(\chi^2\)
  \item
    degrés de liberté égal au nombre de restrictions
  \end{itemize}
\item
  Intervalles de confiance de vraisemblance profilée

  \begin{itemize}
  \tightlist
  \item
    invariants aux reparamétrisations
  \end{itemize}
\end{itemize}
\end{frame}



\end{document}
