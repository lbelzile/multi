% Options for packages loaded elsewhere
\PassOptionsToPackage{unicode}{hyperref}
\PassOptionsToPackage{hyphens}{url}
%
\documentclass[
  ignorenonframetext,
]{beamer}
\usepackage{pgfpages}
\setbeamertemplate{caption}[numbered]
\setbeamertemplate{caption label separator}{: }
\setbeamercolor{caption name}{fg=normal text.fg}
\beamertemplatenavigationsymbolsempty
% Prevent slide breaks in the middle of a paragraph
\widowpenalties 1 10000
\raggedbottom
\setbeamertemplate{part page}{
  \centering
  \begin{beamercolorbox}[sep=16pt,center]{part title}
    \usebeamerfont{part title}\insertpart\par
  \end{beamercolorbox}
}
\setbeamertemplate{section page}{
  \centering
  \begin{beamercolorbox}[sep=12pt,center]{part title}
    \usebeamerfont{section title}\insertsection\par
  \end{beamercolorbox}
}
\setbeamertemplate{subsection page}{
  \centering
  \begin{beamercolorbox}[sep=8pt,center]{part title}
    \usebeamerfont{subsection title}\insertsubsection\par
  \end{beamercolorbox}
}
\AtBeginPart{
  \frame{\partpage}
}
\AtBeginSection{
  \ifbibliography
  \else
    \frame{\sectionpage}
  \fi
}
\AtBeginSubsection{
  \frame{\subsectionpage}
}

\usepackage{amsmath,amssymb}
\usepackage{lmodern}
\usepackage{iftex}
\ifPDFTeX
  \usepackage[T1]{fontenc}
  \usepackage[utf8]{inputenc}
  \usepackage{textcomp} % provide euro and other symbols
\else % if luatex or xetex
  \usepackage{unicode-math}
  \defaultfontfeatures{Scale=MatchLowercase}
  \defaultfontfeatures[\rmfamily]{Ligatures=TeX,Scale=1}
  \setmainfont[]{D-DIN}
\fi
\usecolortheme{Flip}
\usefonttheme{serif} % use mainfont rather than sansfont for slide text
\useinnertheme{Flip}
\useoutertheme{Flip}
% Use upquote if available, for straight quotes in verbatim environments
\IfFileExists{upquote.sty}{\usepackage{upquote}}{}
\IfFileExists{microtype.sty}{% use microtype if available
  \usepackage[]{microtype}
  \UseMicrotypeSet[protrusion]{basicmath} % disable protrusion for tt fonts
}{}
\makeatletter
\@ifundefined{KOMAClassName}{% if non-KOMA class
  \IfFileExists{parskip.sty}{%
    \usepackage{parskip}
  }{% else
    \setlength{\parindent}{0pt}
    \setlength{\parskip}{6pt plus 2pt minus 1pt}}
}{% if KOMA class
  \KOMAoptions{parskip=half}}
\makeatother
\usepackage{xcolor}
\newif\ifbibliography
\setlength{\emergencystretch}{3em} % prevent overfull lines
\setcounter{secnumdepth}{-\maxdimen} % remove section numbering

\usepackage{color}
\usepackage{fancyvrb}
\newcommand{\VerbBar}{|}
\newcommand{\VERB}{\Verb[commandchars=\\\{\}]}
\DefineVerbatimEnvironment{Highlighting}{Verbatim}{commandchars=\\\{\}}
% Add ',fontsize=\small' for more characters per line
\usepackage{framed}
\definecolor{shadecolor}{RGB}{241,243,245}
\newenvironment{Shaded}{\begin{snugshade}}{\end{snugshade}}
\newcommand{\AlertTok}[1]{\textcolor[rgb]{0.68,0.00,0.00}{#1}}
\newcommand{\AnnotationTok}[1]{\textcolor[rgb]{0.37,0.37,0.37}{#1}}
\newcommand{\AttributeTok}[1]{\textcolor[rgb]{0.40,0.45,0.13}{#1}}
\newcommand{\BaseNTok}[1]{\textcolor[rgb]{0.68,0.00,0.00}{#1}}
\newcommand{\BuiltInTok}[1]{\textcolor[rgb]{0.00,0.23,0.31}{#1}}
\newcommand{\CharTok}[1]{\textcolor[rgb]{0.13,0.47,0.30}{#1}}
\newcommand{\CommentTok}[1]{\textcolor[rgb]{0.37,0.37,0.37}{#1}}
\newcommand{\CommentVarTok}[1]{\textcolor[rgb]{0.37,0.37,0.37}{\textit{#1}}}
\newcommand{\ConstantTok}[1]{\textcolor[rgb]{0.56,0.35,0.01}{#1}}
\newcommand{\ControlFlowTok}[1]{\textcolor[rgb]{0.00,0.23,0.31}{#1}}
\newcommand{\DataTypeTok}[1]{\textcolor[rgb]{0.68,0.00,0.00}{#1}}
\newcommand{\DecValTok}[1]{\textcolor[rgb]{0.68,0.00,0.00}{#1}}
\newcommand{\DocumentationTok}[1]{\textcolor[rgb]{0.37,0.37,0.37}{\textit{#1}}}
\newcommand{\ErrorTok}[1]{\textcolor[rgb]{0.68,0.00,0.00}{#1}}
\newcommand{\ExtensionTok}[1]{\textcolor[rgb]{0.00,0.23,0.31}{#1}}
\newcommand{\FloatTok}[1]{\textcolor[rgb]{0.68,0.00,0.00}{#1}}
\newcommand{\FunctionTok}[1]{\textcolor[rgb]{0.28,0.35,0.67}{#1}}
\newcommand{\ImportTok}[1]{\textcolor[rgb]{0.00,0.46,0.62}{#1}}
\newcommand{\InformationTok}[1]{\textcolor[rgb]{0.37,0.37,0.37}{#1}}
\newcommand{\KeywordTok}[1]{\textcolor[rgb]{0.00,0.23,0.31}{#1}}
\newcommand{\NormalTok}[1]{\textcolor[rgb]{0.00,0.23,0.31}{#1}}
\newcommand{\OperatorTok}[1]{\textcolor[rgb]{0.37,0.37,0.37}{#1}}
\newcommand{\OtherTok}[1]{\textcolor[rgb]{0.00,0.23,0.31}{#1}}
\newcommand{\PreprocessorTok}[1]{\textcolor[rgb]{0.68,0.00,0.00}{#1}}
\newcommand{\RegionMarkerTok}[1]{\textcolor[rgb]{0.00,0.23,0.31}{#1}}
\newcommand{\SpecialCharTok}[1]{\textcolor[rgb]{0.37,0.37,0.37}{#1}}
\newcommand{\SpecialStringTok}[1]{\textcolor[rgb]{0.13,0.47,0.30}{#1}}
\newcommand{\StringTok}[1]{\textcolor[rgb]{0.13,0.47,0.30}{#1}}
\newcommand{\VariableTok}[1]{\textcolor[rgb]{0.07,0.07,0.07}{#1}}
\newcommand{\VerbatimStringTok}[1]{\textcolor[rgb]{0.13,0.47,0.30}{#1}}
\newcommand{\WarningTok}[1]{\textcolor[rgb]{0.37,0.37,0.37}{\textit{#1}}}

\providecommand{\tightlist}{%
  \setlength{\itemsep}{0pt}\setlength{\parskip}{0pt}}\usepackage{longtable,booktabs,array}
\usepackage{calc} % for calculating minipage widths
\usepackage{caption}
% Make caption package work with longtable
\makeatletter
\def\fnum@table{\tablename~\thetable}
\makeatother
\usepackage{graphicx}
\makeatletter
\def\maxwidth{\ifdim\Gin@nat@width>\linewidth\linewidth\else\Gin@nat@width\fi}
\def\maxheight{\ifdim\Gin@nat@height>\textheight\textheight\else\Gin@nat@height\fi}
\makeatother
% Scale images if necessary, so that they will not overflow the page
% margins by default, and it is still possible to overwrite the defaults
% using explicit options in \includegraphics[width, height, ...]{}
\setkeys{Gin}{width=\maxwidth,height=\maxheight,keepaspectratio}
% Set default figure placement to htbp
\makeatletter
\def\fps@figure{htbp}
\makeatother

\usepackage{tabu}
\makeatletter
\makeatother
\makeatletter
\makeatother
\makeatletter
\@ifpackageloaded{caption}{}{\usepackage{caption}}
\AtBeginDocument{%
\ifdefined\contentsname
  \renewcommand*\contentsname{Table of contents}
\else
  \newcommand\contentsname{Table of contents}
\fi
\ifdefined\listfigurename
  \renewcommand*\listfigurename{List of Figures}
\else
  \newcommand\listfigurename{List of Figures}
\fi
\ifdefined\listtablename
  \renewcommand*\listtablename{List of Tables}
\else
  \newcommand\listtablename{List of Tables}
\fi
\ifdefined\figurename
  \renewcommand*\figurename{Figure}
\else
  \newcommand\figurename{Figure}
\fi
\ifdefined\tablename
  \renewcommand*\tablename{Table}
\else
  \newcommand\tablename{Table}
\fi
}
\@ifpackageloaded{float}{}{\usepackage{float}}
\floatstyle{ruled}
\@ifundefined{c@chapter}{\newfloat{codelisting}{h}{lop}}{\newfloat{codelisting}{h}{lop}[chapter]}
\floatname{codelisting}{Listing}
\newcommand*\listoflistings{\listof{codelisting}{List of Listings}}
\makeatother
\makeatletter
\@ifpackageloaded{caption}{}{\usepackage{caption}}
\@ifpackageloaded{subcaption}{}{\usepackage{subcaption}}
\makeatother
\makeatletter
\@ifpackageloaded{tcolorbox}{}{\usepackage[many]{tcolorbox}}
\makeatother
\makeatletter
\@ifundefined{shadecolor}{\definecolor{shadecolor}{rgb}{.97, .97, .97}}
\makeatother
\makeatletter
\makeatother
\ifLuaTeX
  \usepackage{selnolig}  % disable illegal ligatures
\fi
\IfFileExists{bookmark.sty}{\usepackage{bookmark}}{\usepackage{hyperref}}
\IfFileExists{xurl.sty}{\usepackage{xurl}}{} % add URL line breaks if available
\urlstyle{same} % disable monospaced font for URLs
\hypersetup{
  pdftitle={Sélection de variables},
  pdfauthor={Léo Belzile},
  hidelinks,
  pdfcreator={LaTeX via pandoc}}

\title{Sélection de variables}
\subtitle{Analyse multidimensionnelle appliquée}
\author{Léo Belzile}
\date{automne 2022}
\institute{HEC Montréal}

\begin{document}
\frame{\titlepage}
\ifdefined\Shaded\renewenvironment{Shaded}{\begin{tcolorbox}[frame hidden, interior hidden, boxrule=0pt, borderline west={3pt}{0pt}{shadecolor}, enhanced, breakable, sharp corners]}{\end{tcolorbox}}\fi

\begin{frame}{Modèles prédictifs}
\protect\hypertarget{moduxe8les-pruxe9dictifs}{}
\textbf{Objectif}: bâtir un modèle pour une variable réponse \(Y\) en
fonction de variables explicatives
\(\mathrm{X}_1, \ldots, \mathrm{X}_p\).

On s'intéresse à
\[\underset{\text{vraie moyenne inconnue}}{f(\mathrm{X}_1, \ldots, \mathrm{X}_p)}.\]

L'analyste détermine
\[\underset{\text{approximation}}{\widehat{f}(\mathrm{X}_1, \ldots, \mathrm{X}_p)},\]
une fonction des variables explicatives.
\end{frame}

\begin{frame}{Rappels sur la régression linéaire}
\protect\hypertarget{rappels-sur-la-ruxe9gression-linuxe9aire}{}
On spécifie que la \textbf{moyenne} de la variable réponse \(Y\) est une
fonction linéaire des variables explicatives
\(\mathrm{X}_1, \ldots, \mathrm{X}_p\), soit

\[\underset{\text{moyenne théorique}}{\mathsf{E}(Y \mid \mathbf{X})} = \underset{\text{
somme pondérée des variables explicatives}}{\beta_0 + \beta_1 \mathrm{X}_{i1} + \cdots + \beta_p \mathrm{X}_{ip}}.\]

en supposant que l'écart entre les observations et cette moyenne est
constant, \[\mathsf{Va}(Y \mid \mathbf{X}) = \sigma^2.\]
\end{frame}

\begin{frame}{Représentation alternative}
\protect\hypertarget{repruxe9sentation-alternative}{}
Pour la \(i\)e observation,

\[\underset{\text{réponse}}{Y_i} = \underset{\text{prédicteur linéaire}}{\beta_0 + \beta_1 \mathrm{X}_{i1} + \cdots + \beta_p \mathrm{X}_{ip}} + \underset{\text{aléa}}{\varepsilon_i}.\]

\begin{itemize}
\tightlist
\item
  L'aléa \(\varepsilon_i\) représente la distance \textbf{verticale}
  entre la vraie pente et l'observation
\item
  Autant d'aléas que d'observations (\(n\)), variable aléatoire
  inconnue\ldots{}
\end{itemize}
\end{frame}

\begin{frame}{Postulats}
\protect\hypertarget{postulats}{}
\begin{itemize}
\tightlist
\item
  L'aléa \(\varepsilon_i\) représente l'erreur, soit la différence entre
  la valeur observée et la moyenne de la population pour les même
  valeurs des variables explicatives.
\item
  On suppose que le modèle pour la moyenne est correctement spécifié:
  l'aléa a une moyenne théorique nulle, \(\mathsf{E}(\varepsilon_i)=0\).
\item
  On suppose que les observations sont indépendantes les unes des
  autres.
\end{itemize}
\end{frame}

\begin{frame}{Régression linéaire en deux dimensions}
\protect\hypertarget{ruxe9gression-linuxe9aire-en-deux-dimensions}{}
Si \(\mathsf{E}(Y)=\beta_0 + \beta_1 \mathrm{X}\), alors

\begin{itemize}
\tightlist
\item
  \(\beta_0\) représente l'ordonnée à l'origine (valeur quand
  \(\mathrm{X}=0\).)
\item
  \(\beta_1\) est la pente
\end{itemize}

\begin{figure}

{\centering \includegraphics[width=0.75\textwidth,height=\textheight]{MATH60602-diapos4_files/figure-beamer/unnamed-chunk-1-1.pdf}

}

\end{figure}
\end{frame}

\begin{frame}{Résidus ordinaires}
\protect\hypertarget{ruxe9sidus-ordinaires}{}
L'estimation des paramètres
\(\widehat{\beta}_0, \cdots, \widehat{\beta}_p\) nous donne
\[\underset{\text{prédiction}}{\widehat{Y}_i} = \widehat{\beta}_0 + \widehat{\beta}_1\mathrm{X}_{i1} \cdots + \widehat{\beta}_p\mathrm{X}_{ip}.\]

On peut approximer l'aléa à l'aide du \textbf{résidu ordinaires}, soit
\[\underset{\text{résidu ordinaire}}{e_i} = \underset{\text{observation}}{Y_i} - \underset{\text{prédiction}}{\widehat{Y}_i}.\]

\begin{itemize}
\tightlist
\item
  par construction, la moyenne des \(e_i\) est zéro.
\item
  le résidu ordinaire est la distance verticale entre l'observation et
  la ``droite'' \textbf{ajustée}
\end{itemize}
\end{frame}

\begin{frame}{Illustration des résidus ordinaires}
\protect\hypertarget{illustration-des-ruxe9sidus-ordinaires}{}
\begin{figure}

{\centering \includegraphics[width=0.8\textwidth,height=\textheight]{MATH60602-diapos4_files/figure-beamer/distancevert-1.pdf}

}

\end{figure}
\end{frame}

\begin{frame}{Erreur quadratique moyenne}
\protect\hypertarget{erreur-quadratique-moyenne}{}
L'erreur quadratique moyenne théorique est

\[\mathsf{E} \left[\left\{ (Y - \widehat{f}(\mathrm{X}_1, \ldots, \mathrm{X}_p)\right\}^2\right],\]
la moyenne de la différence au carré entre la vraie valeur de \(Y\) et
la valeur prédite par le modèle.

En pratique, on remplace la moyenne théorique par une moyenne empirique
obtenue à partir d'un échantillon aléatoire.
\end{frame}

\begin{frame}{Estimation des paramètres}
\protect\hypertarget{estimation-des-paramuxe8tres}{}
Comment estimer les paramètres \(\beta_0, \ldots, \beta_p\)?

\textbf{Optimisation}: trouver les valeurs qui minimisent l'erreur
quadratique moyenne \textbf{empirique} avec l'échantillon des \(n\)
observations, soit

\[\frac{e_1^2 + \cdots + e_n^2}{n}\] Solution explicite
\(\widehat{\boldsymbol{\beta}} = (\mathbf{X}^\top\mathbf{X})^{-1}\mathbf{X}^\top\boldsymbol{Y}\)!
\end{frame}

\begin{frame}[fragile]{Estimation dans \textbf{R}}
\protect\hypertarget{estimation-dans-r}{}
La fonction \texttt{lm} calcule l'ajustement du modèle linéaire.

Arguments:

\begin{itemize}
\tightlist
\item
  \texttt{formula}: formule de type
  \texttt{reponse\ \textasciitilde{}\ \ variables\ explicatives}, où les
  variables explicatives sont séparées par un signe \texttt{+}
\item
  \texttt{data}: base de données
\end{itemize}

\begin{Shaded}
\begin{Highlighting}[numbers=left,,]
\NormalTok{modlin }\OtherTok{\textless{}{-}} \FunctionTok{lm}\NormalTok{(mpg }\SpecialCharTok{\textasciitilde{}}\NormalTok{ hp }\SpecialCharTok{+}\NormalTok{ wt, }
             \AttributeTok{data =}\NormalTok{ mtcars)}
\FunctionTok{summary}\NormalTok{(modlin)}
\end{Highlighting}
\end{Shaded}
\end{frame}

\begin{frame}[fragile]{Sortie de \texttt{summary}}
\protect\hypertarget{sortie-de-summary}{}
\footnotesize

\begin{verbatim}

Call:
lm(formula = mpg ~ hp + wt, data = mtcars)

Residuals:
   Min     1Q Median     3Q    Max 
-3.941 -1.600 -0.182  1.050  5.854 

Coefficients:
            Estimate Std. Error t value Pr(>|t|)    
(Intercept) 37.22727    1.59879  23.285  < 2e-16 ***
hp          -0.03177    0.00903  -3.519  0.00145 ** 
wt          -3.87783    0.63273  -6.129 1.12e-06 ***
---
Signif. codes:  0 '***' 0.001 '**' 0.01 '*' 0.05 '.' 0.1 ' ' 1

Residual standard error: 2.593 on 29 degrees of freedom
Multiple R-squared:  0.8268,    Adjusted R-squared:  0.8148 
F-statistic: 69.21 on 2 and 29 DF,  p-value: 9.109e-12
\end{verbatim}

\normalsize
\end{frame}

\begin{frame}{Tableau de sortie}
\protect\hypertarget{tableau-de-sortie}{}
\begin{itemize}
\tightlist
\item
  Formule de l'appel
\item
  Statistiques descriptives des résidus ordinaires \(e_1, \ldots, e_n\).
\item
  Tableau des estimations

  \begin{itemize}
  \tightlist
  \item
    Coefficients \(\widehat{\beta}_j\)
  \item
    Erreurs-types, \(\mathsf{se}(\widehat{\beta}_j)\)
  \item
    Statistique du test-\emph{t} pour \(\mathscr{H}_0: \beta_j=0\), soit
    \(t=\widehat{\beta}_j/\mathsf{se}(\widehat{\beta}_j)\)
  \item
    Valeur-\emph{p} selon loi nulle \(\mathsf{St}(n-p-1)\)
  \end{itemize}
\item
  Estimation de l'écart-type \(\widehat{\sigma}\) et degrés de liberté
  \(n-p-1\)
\item
  Estimations du coefficient de détermination, \(R^2\) et \(R^2\) ajusté
\item
  Statistique \(F\) d'ajustement global et valeur-\(p\) de
  \(\mathsf{F}(p, n - p - 1)\)

  \begin{itemize}
  \tightlist
  \item
    \(\mathscr{H}_a\): modèle linéaire
  \item
    \(\mathscr{H}_0\): modèle avec uniquement ordonnée à l'origine
    (chaque observation prédite par la moyenne des réponses,
    \(\overline{Y}\))
  \end{itemize}
\end{itemize}

\normalsize
\end{frame}

\begin{frame}[fragile]{Quelques méthodes pour \texttt{lm}}
\protect\hypertarget{quelques-muxe9thodes-pour-lm}{}
\begin{itemize}
\tightlist
\item
  \texttt{resid} pour les résidus ordinaires \(e_i\)
\item
  \texttt{fitted} pour les valeurs ajustées \(\widehat{Y}_i\)
\item
  \texttt{coef} pour les estimations des paramètres
  \(\widehat{\beta}_0, \ldots, \widehat{\beta}_p\)
\item
  \texttt{plot} pour des diagnostics graphiques d'ajustement
\item
  \texttt{anova} pour la comparaison de modèles emboîtés
\item
  \texttt{predict} pour les prédictions (avec nouvelles données)
\item
  \texttt{confint} pour intervalles de confiance pour les paramètres.
\end{itemize}
\end{frame}

\begin{frame}[fragile]{Variables catégorielles}
\protect\hypertarget{variables-catuxe9gorielles}{}
\begin{itemize}
\tightlist
\item
  Les facteurs (\texttt{\textless{}factor\textgreater{}}) sont traités
  adéquatement par \textbf{R}.
\item
  Si la variable a \(K\) valeurs possibles (niveaux), le modèle inclut
  \(K-1\) indicatrices 0/1.
\item
  Par défaut dans \textbf{\$}, la catégorie de référence est la plus
  petite en ordre alphanumérique.
\end{itemize}
\end{frame}

\begin{frame}[fragile]{Encodage des variables catégorielles}
\protect\hypertarget{encodage-des-variables-catuxe9gorielles}{}
Considérons une variable catégorielle \texttt{cat} avec niveaux
\texttt{1}, \texttt{2}, et \texttt{3}.

\begin{longtable}[]{@{}ccc@{}}
\toprule()
\texttt{cat} & \texttt{cat2} & \texttt{cat3} \\
\midrule()
\endhead
1 & 0 & 0 \\
2 & 1 & 0 \\
3 & 0 & 1 \\
\bottomrule()
\end{longtable}

La catégorie de référence est associée à l'ordonnée à l'origine (quand
\texttt{cat2=0} et \texttt{cat3=0}).
\end{frame}

\begin{frame}{Prédiction vs inférence}
\protect\hypertarget{pruxe9diction-vs-infuxe9rence}{}
\end{frame}

\begin{frame}[fragile]{Sélection de variables}
\protect\hypertarget{suxe9lection-de-variables}{}
\begin{itemize}
\tightlist
\item
  Comment choisir quelles variables inclure?
\item
  Quel est la spécification adéquate pour
  \(f(\mathrm{X}_1, \ldots, \mathrm{X}_p)\)?

  \begin{itemize}
  \tightlist
  \item
    régression, réseaux de neurone, forêts aléatoires, etc.
  \item
    transformations de variables, \texttt{age}\({}^2\),
    \(\ln(\)\texttt{age}\()\), etc.
  \end{itemize}
\end{itemize}

Notre but sera de sélectionner un \textbf{bon} modèle, selon les
objectifs de l'étude
\end{frame}

\begin{frame}{}
\protect\hypertarget{section}{}
\end{frame}

\begin{frame}{Illustration du surajustement}
\protect\hypertarget{illustration-du-surajustement}{}
\begin{figure}

{\centering \includegraphics[width=0.7\textwidth,height=\textheight]{MATH60602-diapos4_files/figure-beamer/overfitting-1.pdf}

}

\end{figure}
\end{frame}

\begin{frame}
\begin{block}{Séparation des données}
\protect\hypertarget{suxe9paration-des-donnuxe9es}{}
Ne pas utiliser les données employés pour ajuster un modèle pour
\textbf{prédire la performance}

\begin{itemize}
\tightlist
\item
  échantillons d'apprentissage/validation/test (fixes)
\item
  validation croisée (avec \(K=5, 10\) groupes), mais \emph{résultat
  aléatoire}
\end{itemize}
\end{block}
\end{frame}

\begin{frame}
\end{frame}



\end{document}
