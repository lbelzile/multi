% Options for packages loaded elsewhere
\PassOptionsToPackage{unicode}{hyperref}
\PassOptionsToPackage{hyphens}{url}
%
\documentclass[
  ignorenonframetext,
]{beamer}
\usepackage{pgfpages}
\setbeamertemplate{caption}[numbered]
\setbeamertemplate{caption label separator}{: }
\setbeamercolor{caption name}{fg=normal text.fg}
\beamertemplatenavigationsymbolsempty
% Prevent slide breaks in the middle of a paragraph
\widowpenalties 1 10000
\raggedbottom
\setbeamertemplate{part page}{
  \centering
  \begin{beamercolorbox}[sep=16pt,center]{part title}
    \usebeamerfont{part title}\insertpart\par
  \end{beamercolorbox}
}
\setbeamertemplate{section page}{
  \centering
  \begin{beamercolorbox}[sep=12pt,center]{part title}
    \usebeamerfont{section title}\insertsection\par
  \end{beamercolorbox}
}
\setbeamertemplate{subsection page}{
  \centering
  \begin{beamercolorbox}[sep=8pt,center]{part title}
    \usebeamerfont{subsection title}\insertsubsection\par
  \end{beamercolorbox}
}
\AtBeginPart{
  \frame{\partpage}
}
\AtBeginSection{
  \ifbibliography
  \else
    \frame{\sectionpage}
  \fi
}
\AtBeginSubsection{
  \frame{\subsectionpage}
}

\usepackage{amsmath,amssymb}
\usepackage{lmodern}
\usepackage{iftex}
\ifPDFTeX
  \usepackage[T1]{fontenc}
  \usepackage[utf8]{inputenc}
  \usepackage{textcomp} % provide euro and other symbols
\else % if luatex or xetex
  \usepackage{unicode-math}
  \defaultfontfeatures{Scale=MatchLowercase}
  \defaultfontfeatures[\rmfamily]{Ligatures=TeX,Scale=1}
  \setmainfont[]{D-DIN}
  \setsansfont[]{Latin Modern Sans}
  \setmathfont[]{Latin Modern Math}
\fi
\usecolortheme{Flip}
\usefonttheme{serif} % use mainfont rather than sansfont for slide text
\useinnertheme{Flip}
\useoutertheme{Flip}
% Use upquote if available, for straight quotes in verbatim environments
\IfFileExists{upquote.sty}{\usepackage{upquote}}{}
\IfFileExists{microtype.sty}{% use microtype if available
  \usepackage[]{microtype}
  \UseMicrotypeSet[protrusion]{basicmath} % disable protrusion for tt fonts
}{}
\makeatletter
\@ifundefined{KOMAClassName}{% if non-KOMA class
  \IfFileExists{parskip.sty}{%
    \usepackage{parskip}
  }{% else
    \setlength{\parindent}{0pt}
    \setlength{\parskip}{6pt plus 2pt minus 1pt}}
}{% if KOMA class
  \KOMAoptions{parskip=half}}
\makeatother
\usepackage{xcolor}
\newif\ifbibliography
\setlength{\emergencystretch}{3em} % prevent overfull lines
\setcounter{secnumdepth}{-\maxdimen} % remove section numbering

\usepackage{color}
\usepackage{fancyvrb}
\newcommand{\VerbBar}{|}
\newcommand{\VERB}{\Verb[commandchars=\\\{\}]}
\DefineVerbatimEnvironment{Highlighting}{Verbatim}{commandchars=\\\{\}}
% Add ',fontsize=\small' for more characters per line
\usepackage{framed}
\definecolor{shadecolor}{RGB}{241,243,245}
\newenvironment{Shaded}{\begin{snugshade}}{\end{snugshade}}
\newcommand{\AlertTok}[1]{\textcolor[rgb]{0.68,0.00,0.00}{#1}}
\newcommand{\AnnotationTok}[1]{\textcolor[rgb]{0.37,0.37,0.37}{#1}}
\newcommand{\AttributeTok}[1]{\textcolor[rgb]{0.40,0.45,0.13}{#1}}
\newcommand{\BaseNTok}[1]{\textcolor[rgb]{0.68,0.00,0.00}{#1}}
\newcommand{\BuiltInTok}[1]{\textcolor[rgb]{0.00,0.23,0.31}{#1}}
\newcommand{\CharTok}[1]{\textcolor[rgb]{0.13,0.47,0.30}{#1}}
\newcommand{\CommentTok}[1]{\textcolor[rgb]{0.37,0.37,0.37}{#1}}
\newcommand{\CommentVarTok}[1]{\textcolor[rgb]{0.37,0.37,0.37}{\textit{#1}}}
\newcommand{\ConstantTok}[1]{\textcolor[rgb]{0.56,0.35,0.01}{#1}}
\newcommand{\ControlFlowTok}[1]{\textcolor[rgb]{0.00,0.23,0.31}{#1}}
\newcommand{\DataTypeTok}[1]{\textcolor[rgb]{0.68,0.00,0.00}{#1}}
\newcommand{\DecValTok}[1]{\textcolor[rgb]{0.68,0.00,0.00}{#1}}
\newcommand{\DocumentationTok}[1]{\textcolor[rgb]{0.37,0.37,0.37}{\textit{#1}}}
\newcommand{\ErrorTok}[1]{\textcolor[rgb]{0.68,0.00,0.00}{#1}}
\newcommand{\ExtensionTok}[1]{\textcolor[rgb]{0.00,0.23,0.31}{#1}}
\newcommand{\FloatTok}[1]{\textcolor[rgb]{0.68,0.00,0.00}{#1}}
\newcommand{\FunctionTok}[1]{\textcolor[rgb]{0.28,0.35,0.67}{#1}}
\newcommand{\ImportTok}[1]{\textcolor[rgb]{0.00,0.46,0.62}{#1}}
\newcommand{\InformationTok}[1]{\textcolor[rgb]{0.37,0.37,0.37}{#1}}
\newcommand{\KeywordTok}[1]{\textcolor[rgb]{0.00,0.23,0.31}{#1}}
\newcommand{\NormalTok}[1]{\textcolor[rgb]{0.00,0.23,0.31}{#1}}
\newcommand{\OperatorTok}[1]{\textcolor[rgb]{0.37,0.37,0.37}{#1}}
\newcommand{\OtherTok}[1]{\textcolor[rgb]{0.00,0.23,0.31}{#1}}
\newcommand{\PreprocessorTok}[1]{\textcolor[rgb]{0.68,0.00,0.00}{#1}}
\newcommand{\RegionMarkerTok}[1]{\textcolor[rgb]{0.00,0.23,0.31}{#1}}
\newcommand{\SpecialCharTok}[1]{\textcolor[rgb]{0.37,0.37,0.37}{#1}}
\newcommand{\SpecialStringTok}[1]{\textcolor[rgb]{0.13,0.47,0.30}{#1}}
\newcommand{\StringTok}[1]{\textcolor[rgb]{0.13,0.47,0.30}{#1}}
\newcommand{\VariableTok}[1]{\textcolor[rgb]{0.07,0.07,0.07}{#1}}
\newcommand{\VerbatimStringTok}[1]{\textcolor[rgb]{0.13,0.47,0.30}{#1}}
\newcommand{\WarningTok}[1]{\textcolor[rgb]{0.37,0.37,0.37}{\textit{#1}}}

\providecommand{\tightlist}{%
  \setlength{\itemsep}{0pt}\setlength{\parskip}{0pt}}\usepackage{longtable,booktabs,array}
\usepackage{calc} % for calculating minipage widths
\usepackage{caption}
% Make caption package work with longtable
\makeatletter
\def\fnum@table{\tablename~\thetable}
\makeatother
\usepackage{graphicx}
\makeatletter
\def\maxwidth{\ifdim\Gin@nat@width>\linewidth\linewidth\else\Gin@nat@width\fi}
\def\maxheight{\ifdim\Gin@nat@height>\textheight\textheight\else\Gin@nat@height\fi}
\makeatother
% Scale images if necessary, so that they will not overflow the page
% margins by default, and it is still possible to overwrite the defaults
% using explicit options in \includegraphics[width, height, ...]{}
\setkeys{Gin}{width=\maxwidth,height=\maxheight,keepaspectratio}
% Set default figure placement to htbp
\makeatletter
\def\fps@figure{htbp}
\makeatother

\usepackage{booktabs}
\usepackage{longtable}
\usepackage{array}
\usepackage{multirow}
\usepackage{wrapfig}
\usepackage{float}
\usepackage{colortbl}
\usepackage{pdflscape}
\usepackage{tabu}
\usepackage{threeparttable}
\usepackage{threeparttablex}
\usepackage[normalem]{ulem}
\usepackage{makecell}
\usepackage{xcolor}
\usepackage{tabu}
\usepackage{mathtools}
\usepackage{mathrsfs}
\makeatletter
\makeatother
\makeatletter
\makeatother
\makeatletter
\@ifpackageloaded{caption}{}{\usepackage{caption}}
\AtBeginDocument{%
\ifdefined\contentsname
  \renewcommand*\contentsname{Table of contents}
\else
  \newcommand\contentsname{Table of contents}
\fi
\ifdefined\listfigurename
  \renewcommand*\listfigurename{List of Figures}
\else
  \newcommand\listfigurename{List of Figures}
\fi
\ifdefined\listtablename
  \renewcommand*\listtablename{List of Tables}
\else
  \newcommand\listtablename{List of Tables}
\fi
\ifdefined\figurename
  \renewcommand*\figurename{Figure}
\else
  \newcommand\figurename{Figure}
\fi
\ifdefined\tablename
  \renewcommand*\tablename{Table}
\else
  \newcommand\tablename{Table}
\fi
}
\@ifpackageloaded{float}{}{\usepackage{float}}
\floatstyle{ruled}
\@ifundefined{c@chapter}{\newfloat{codelisting}{h}{lop}}{\newfloat{codelisting}{h}{lop}[chapter]}
\floatname{codelisting}{Listing}
\newcommand*\listoflistings{\listof{codelisting}{List of Listings}}
\makeatother
\makeatletter
\@ifpackageloaded{caption}{}{\usepackage{caption}}
\@ifpackageloaded{subcaption}{}{\usepackage{subcaption}}
\makeatother
\makeatletter
\@ifpackageloaded{tcolorbox}{}{\usepackage[many]{tcolorbox}}
\makeatother
\makeatletter
\@ifundefined{shadecolor}{\definecolor{shadecolor}{rgb}{.97, .97, .97}}
\makeatother
\makeatletter
\makeatother
\ifLuaTeX
  \usepackage{selnolig}  % disable illegal ligatures
\fi
\IfFileExists{bookmark.sty}{\usepackage{bookmark}}{\usepackage{hyperref}}
\IfFileExists{xurl.sty}{\usepackage{xurl}}{} % add URL line breaks if available
\urlstyle{same} % disable monospaced font for URLs
\hypersetup{
  pdftitle={Analyse de survie, analyse de regroupements},
  pdfauthor={Léo Belzile},
  hidelinks,
  pdfcreator={LaTeX via pandoc}}

\title{Analyse de survie, analyse de regroupements}
\subtitle{Analyse multidimensionnelle appliquée}
\author{Léo Belzile}
\date{automne 2022}
\institute{HEC Montréal}

\begin{document}
\frame{\titlepage}
\ifdefined\Shaded\renewenvironment{Shaded}{\begin{tcolorbox}[enhanced, borderline west={3pt}{0pt}{shadecolor}, boxrule=0pt, breakable, sharp corners, interior hidden, frame hidden]}{\end{tcolorbox}}\fi

\begin{frame}{Modèle à risques proportionnels de Cox}
\protect\hypertarget{moduxe8le-uxe0-risques-proportionnels-de-cox}{}
Le \textbf{modèle à risques proportionnels de Cox} pour \(\mathbf{X}\)
au temps \(t\) est \begin{align*}
h(t; \mathbf{X}) = h_0(t)\exp(\beta_1\mathrm{X}_1 + \cdots + \beta_p \mathrm{X}_p),
\end{align*} où \(h_0(t)\) est la fonction de risque de base qui
remplace l'ordonnée à l'origine.

\begin{itemize}
\tightlist
\item
  Postulat de risques proportionnels: le rapport de risque pour deux
  observations ne varie pas en fonction du temps \(t\).
\end{itemize}
\end{frame}

\begin{frame}{Postulat de risques proportionnels}
\protect\hypertarget{postulat-de-risques-proportionnels}{}
\begin{figure}

{\centering \includegraphics[width=0.9\textwidth,height=\textheight]{MATH60602-diapos10_files/figure-beamer/fig-risquepropfig-1.pdf}

}

\caption{\label{fig-risquepropfig}Courbes de risques proportionnelles
(panneau supérieur) et non proportionnelles (panneau inférieur).}

\end{figure}
\end{frame}

\begin{frame}{Absence de proportionnalité et stratification}
\protect\hypertarget{absence-de-proportionnalituxe9-et-stratification}{}
On peut modéliser la non-proportionnalité par la \textbf{stratification}
pour une variable catégorielle \(Z=1, \ldots, K\).

Supposons que l'effet de \(Z\) sur le risque varie dans le temps.

On écrit alors \begin{align*}
h(t; \mathbf{X}, Z=k) = h_k(t)\exp(\beta_1\mathrm{X}_1 + \cdots + \beta_p \mathrm{X}_p),
\end{align*} où \(h_k\) est la fonction de risque pour \(Z=k\).

Dans ce modèle

\begin{itemize}
\tightlist
\item
  On suppose que l'effet des variables explicatives \(\mathbf{X}\) est
  le même peut importe la valeur de \(Z\).
\item
  L'effet de \(Z=k\) vs \(Z=j\) pour un même ensemble de variables
  explicatives \(\mathbf{X}\) est \(h_k(t)/h_j(t)\), qui dépend du
  temps.
\end{itemize}
\end{frame}

\begin{frame}{Stratification}
\protect\hypertarget{stratification}{}
\begin{itemize}
\tightlist
\item
  \textbf{Avantage}: on peut modéliser n'importe quel changement du
  risque en fonction de \(Z\).
\item
  \textbf{Désavantage}: on perd la variable explicative \(Z\), donc on
  ne peut tester son effet (pas de coefficient)\ldots{} on peut résumer
  l'information pour la variable \(Z\) en calculant par exemple les
  différences de survie à des temps donnés.
\item
  \textbf{Désavantage}: la fonction de risque est estimée pour chaque
  sous-groupe de \(Z\) (plus faible taille d'échantillon).
\end{itemize}

Idéalement, utiliser la stratification avec des variables secondaires ou
de contrôles.
\end{frame}

\begin{frame}[fragile]{Modèle de Cox avec stratification dans
\textbf{R}}
\protect\hypertarget{moduxe8le-de-cox-avec-stratification-dans-r}{}
\begin{Shaded}
\begin{Highlighting}[numbers=left,,]
\FunctionTok{library}\NormalTok{(survival)}
\FunctionTok{data}\NormalTok{(survie1, }\AttributeTok{package =} \StringTok{"hecmulti"}\NormalTok{)}
\CommentTok{\# Stratification par service}
\NormalTok{cox\_strat }\OtherTok{\textless{}{-}} \FunctionTok{coxph}\NormalTok{(}
  \FunctionTok{Surv}\NormalTok{(temps, }\DecValTok{1}\SpecialCharTok{{-}}\NormalTok{censure) }\SpecialCharTok{\textasciitilde{}}\NormalTok{ age }\SpecialCharTok{+}\NormalTok{ sexe }\SpecialCharTok{+} \FunctionTok{strata}\NormalTok{(service), }
  \AttributeTok{data =}\NormalTok{ survie1)}
\CommentTok{\# Décompte par service}
\FunctionTok{with}\NormalTok{(survie1, }\FunctionTok{table}\NormalTok{(service))}
\CommentTok{\# Coefficients}
\FunctionTok{summary}\NormalTok{(cox\_strat)}
\end{Highlighting}
\end{Shaded}
\end{frame}

\begin{frame}{Sorties}
\protect\hypertarget{sorties}{}
\hypertarget{tbl-nserv}{}
\begin{table}
\caption{\label{tbl-nserv}Décompte du nombre d'observations par service. }\tabularnewline

\centering
\begin{tabular}{lrrrr}
\toprule
  & 0 & 1 & 2 & 3\\
\midrule
0 & 197 & 179 & 78 & 46\\
\bottomrule
\end{tabular}
\end{table}

\hypertarget{tbl-coxstratif}{}
\begin{table}
\caption{\label{tbl-coxstratif}Rapport de risques pour un modèle de Cox stratifié par service. }\tabularnewline

\centering
\begin{tabular}{lrrr}
\toprule
terme & exp(coef) & borne inf. & borne sup.\\
\midrule
age & 0.96 & 0.94 & 0.97\\
sexe & 0.61 & 0.44 & 0.85\\
\bottomrule
\end{tabular}
\end{table}
\end{frame}

\begin{frame}{Courbes de survie du modèle stratifié}
\protect\hypertarget{courbes-de-survie-du-moduxe8le-stratifiuxe9}{}
\begin{figure}

{\centering \includegraphics[width=0.85\textwidth,height=\textheight]{MATH60602-diapos10_files/figure-beamer/unnamed-chunk-6-1.pdf}

}

\end{figure}
\end{frame}

\begin{frame}{Extensions du modèle de Cox}
\protect\hypertarget{extensions-du-moduxe8le-de-cox}{}
\begin{enumerate}
\tightlist
\item
  Inclusion de variables explicatives dont la valeur change dans le
  temps.
\item
  Modèle à risques compétitifs.
\end{enumerate}
\end{frame}

\begin{frame}{Évolution temporelle de variables explicatives}
\protect\hypertarget{uxe9volution-temporelle-de-variables-explicatives}{}
On considère une extension du modèle de Cox qui permet d'inclure des
variables explicatives dont la valeur change dans le temps.
\end{frame}



\end{document}
