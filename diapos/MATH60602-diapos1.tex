% Options for packages loaded elsewhere
\PassOptionsToPackage{unicode}{hyperref}
\PassOptionsToPackage{hyphens}{url}
%
\documentclass[
  ignorenonframetext,
]{beamer}
\usepackage{pgfpages}
\setbeamertemplate{caption}[numbered]
\setbeamertemplate{caption label separator}{: }
\setbeamercolor{caption name}{fg=normal text.fg}
\beamertemplatenavigationsymbolsempty
% Prevent slide breaks in the middle of a paragraph
\widowpenalties 1 10000
\raggedbottom
\setbeamertemplate{part page}{
  \centering
  \begin{beamercolorbox}[sep=16pt,center]{part title}
    \usebeamerfont{part title}\insertpart\par
  \end{beamercolorbox}
}
\setbeamertemplate{section page}{
  \centering
  \begin{beamercolorbox}[sep=12pt,center]{part title}
    \usebeamerfont{section title}\insertsection\par
  \end{beamercolorbox}
}
\setbeamertemplate{subsection page}{
  \centering
  \begin{beamercolorbox}[sep=8pt,center]{part title}
    \usebeamerfont{subsection title}\insertsubsection\par
  \end{beamercolorbox}
}
\AtBeginPart{
  \frame{\partpage}
}
\AtBeginSection{
  \ifbibliography
  \else
    \frame{\sectionpage}
  \fi
}
\AtBeginSubsection{
  \frame{\subsectionpage}
}

\usepackage{amsmath,amssymb}
\usepackage{lmodern}
\usepackage{iftex}
\ifPDFTeX
  \usepackage[T1]{fontenc}
  \usepackage[utf8]{inputenc}
  \usepackage{textcomp} % provide euro and other symbols
\else % if luatex or xetex
  \usepackage{unicode-math}
  \defaultfontfeatures{Scale=MatchLowercase}
  \defaultfontfeatures[\rmfamily]{Ligatures=TeX,Scale=1}
  \setmainfont[]{D-DIN}
\fi
\usecolortheme{Flip}
\usefonttheme{serif} % use mainfont rather than sansfont for slide text
\useinnertheme{Flip}
\useoutertheme{Flip}
% Use upquote if available, for straight quotes in verbatim environments
\IfFileExists{upquote.sty}{\usepackage{upquote}}{}
\IfFileExists{microtype.sty}{% use microtype if available
  \usepackage[]{microtype}
  \UseMicrotypeSet[protrusion]{basicmath} % disable protrusion for tt fonts
}{}
\makeatletter
\@ifundefined{KOMAClassName}{% if non-KOMA class
  \IfFileExists{parskip.sty}{%
    \usepackage{parskip}
  }{% else
    \setlength{\parindent}{0pt}
    \setlength{\parskip}{6pt plus 2pt minus 1pt}}
}{% if KOMA class
  \KOMAoptions{parskip=half}}
\makeatother
\usepackage{xcolor}
\newif\ifbibliography
\setlength{\emergencystretch}{3em} % prevent overfull lines
\setcounter{secnumdepth}{-\maxdimen} % remove section numbering


\providecommand{\tightlist}{%
  \setlength{\itemsep}{0pt}\setlength{\parskip}{0pt}}\usepackage{longtable,booktabs,array}
\usepackage{calc} % for calculating minipage widths
\usepackage{caption}
% Make caption package work with longtable
\makeatletter
\def\fnum@table{\tablename~\thetable}
\makeatother
\usepackage{graphicx}
\makeatletter
\def\maxwidth{\ifdim\Gin@nat@width>\linewidth\linewidth\else\Gin@nat@width\fi}
\def\maxheight{\ifdim\Gin@nat@height>\textheight\textheight\else\Gin@nat@height\fi}
\makeatother
% Scale images if necessary, so that they will not overflow the page
% margins by default, and it is still possible to overwrite the defaults
% using explicit options in \includegraphics[width, height, ...]{}
\setkeys{Gin}{width=\maxwidth,height=\maxheight,keepaspectratio}
% Set default figure placement to htbp
\makeatletter
\def\fps@figure{htbp}
\makeatother

\usepackage{tabu}
\makeatletter
\makeatother
\makeatletter
\makeatother
\makeatletter
\@ifpackageloaded{caption}{}{\usepackage{caption}}
\AtBeginDocument{%
\ifdefined\contentsname
  \renewcommand*\contentsname{Table des matières}
\else
  \newcommand\contentsname{Table des matières}
\fi
\ifdefined\listfigurename
  \renewcommand*\listfigurename{Liste des figures}
\else
  \newcommand\listfigurename{Liste des figures}
\fi
\ifdefined\listtablename
  \renewcommand*\listtablename{Liste des tableaux}
\else
  \newcommand\listtablename{Liste des tableaux}
\fi
\ifdefined\figurename
  \renewcommand*\figurename{Figure}
\else
  \newcommand\figurename{Figure}
\fi
\ifdefined\tablename
  \renewcommand*\tablename{Tableau}
\else
  \newcommand\tablename{Tableau}
\fi
}
\@ifpackageloaded{float}{}{\usepackage{float}}
\floatstyle{ruled}
\@ifundefined{c@chapter}{\newfloat{codelisting}{h}{lop}}{\newfloat{codelisting}{h}{lop}[chapter]}
\floatname{codelisting}{Énumération}
\newcommand*\listoflistings{\listof{codelisting}{Liste des énumérations}}
\makeatother
\makeatletter
\@ifpackageloaded{caption}{}{\usepackage{caption}}
\@ifpackageloaded{subcaption}{}{\usepackage{subcaption}}
\makeatother
\makeatletter
\@ifpackageloaded{tcolorbox}{}{\usepackage[many]{tcolorbox}}
\makeatother
\makeatletter
\@ifundefined{shadecolor}{\definecolor{shadecolor}{rgb}{.97, .97, .97}}
\makeatother
\makeatletter
\makeatother
\makeatletter
\@ifpackageloaded{fontawesome5}{}{\usepackage{fontawesome5}}
\makeatother
\ifLuaTeX
  \usepackage{selnolig}  % disable illegal ligatures
\fi
\IfFileExists{bookmark.sty}{\usepackage{bookmark}}{\usepackage{hyperref}}
\IfFileExists{xurl.sty}{\usepackage{xurl}}{} % add URL line breaks if available
\urlstyle{same} % disable monospaced font for URLs
\hypersetup{
  pdftitle={Analyse exploratoire},
  pdfauthor={Léo Belzile},
  hidelinks,
  pdfcreator={LaTeX via pandoc}}

\title{Analyse exploratoire}
\subtitle{Analyse multidimensionnelle appliquée}
\author{Léo Belzile}
\date{automne 2022}
\institute{HEC Montréal}

\begin{document}
\frame{\titlepage}
\ifdefined\Shaded\renewenvironment{Shaded}{\begin{tcolorbox}[boxrule=0pt, interior hidden, sharp corners, borderline west={3pt}{0pt}{shadecolor}, enhanced, frame hidden, breakable]}{\end{tcolorbox}}\fi

\begin{frame}{Notions de base}
\protect\hypertarget{notions-de-base}{}
\begin{itemize}
\tightlist
\item
  Variable
\item
  Observation
\end{itemize}

\textbf{Bonnes pratiques pour l'organisation de données:}

\begin{quote}
Karl W. Broman \& Kara H. Woo (2018) Data Organization in Spreadsheets,
The American Statistician, 72:1, 2-10, DOI:
10.1080/00031305.2017.1375989
\end{quote}
\end{frame}

\begin{frame}{Types de variables numériques}
\protect\hypertarget{types-de-variables-numuxe9riques}{}
\begin{figure}

{\centering \includegraphics[width=0.8\textwidth,height=\textheight]{figures/continuous_discrete.png}

}

\caption{Allison Horst (CC BY 4.0)}

\end{figure}
\end{frame}

\begin{frame}{Types de variables catégorielles}
\protect\hypertarget{types-de-variables-catuxe9gorielles}{}
\begin{figure}

{\centering \includegraphics[width=0.8\textwidth,height=\textheight]{figures/nominal_ordinal_binary.png}

}

\caption{IAllison Horst (CC BY 4.0)}

\end{figure}
\end{frame}

\begin{frame}{Données en format ``tidy''}
\protect\hypertarget{donnuxe9es-en-format-tidy}{}
\begin{itemize}
\tightlist
\item
  variables en colonnes
\item
  observations en lignes
\item
  une seule mesure par cellule
\end{itemize}

\begin{figure}

{\centering \includegraphics[width=0.7\textwidth,height=\textheight]{figures/tidydata_1.jpg}

}

\caption{Allison Horst (CC BY 4.0)}

\end{figure}
\end{frame}

\begin{frame}[fragile]{Validation des données}
\protect\hypertarget{validation-des-donnuxe9es}{}
Vérifier la présence de

\begin{itemize}
\tightlist
\item
  valeurs manquantes (\texttt{NA}), \(-999\), etc.
\item
  relations logiques (total, moyenne, etc.) entre variables
\item
  variables catégorielles non déclarées

  \begin{itemize}
  \tightlist
  \item
    valeur entière
  \item
    chaînes de caractère
  \end{itemize}
\end{itemize}
\end{frame}

\begin{frame}{Analyse exploratoire}
\protect\hypertarget{analyse-exploratoire}{}
\begin{enumerate}
\tightlist
\item
  Formuler des questions
\item
  Chercher des réponses à ces questions

  \begin{itemize}
  \tightlist
  \item
    statistiques descriptives
  \item
    tableaux de contingence
  \item
    graphiques
  \end{itemize}
\item
  Raffiner les questions
\end{enumerate}
\end{frame}

\begin{frame}{Représentation graphique de base}
\protect\hypertarget{repruxe9sentation-graphique-de-base}{}
Passer en revue les types de graphiques selon les types de variable
\end{frame}

\begin{frame}{Éléments graphiques clés}
\protect\hypertarget{uxe9luxe9ments-graphiques-cluxe9s}{}
\begin{itemize}
\tightlist
\item
  Titre et annotation
\item
  Libellés et unités sur les axes
\item
  Libellé de l'axe des \(y\) en sous-titre
\item
  Rotation si étiquettes trop longue (variable discrète)
\item
  Palette de couleur pour daltoniens
\item
  Taille de police suffisante pour lisibilité!
\end{itemize}
\end{frame}



\end{document}
