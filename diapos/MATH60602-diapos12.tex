% Options for packages loaded elsewhere
\PassOptionsToPackage{unicode}{hyperref}
\PassOptionsToPackage{hyphens}{url}
%
\documentclass[
  ignorenonframetext,
]{beamer}
\usepackage{pgfpages}
\setbeamertemplate{caption}[numbered]
\setbeamertemplate{caption label separator}{: }
\setbeamercolor{caption name}{fg=normal text.fg}
\beamertemplatenavigationsymbolsempty
% Prevent slide breaks in the middle of a paragraph
\widowpenalties 1 10000
\raggedbottom
\setbeamertemplate{part page}{
  \centering
  \begin{beamercolorbox}[sep=16pt,center]{part title}
    \usebeamerfont{part title}\insertpart\par
  \end{beamercolorbox}
}
\setbeamertemplate{section page}{
  \centering
  \begin{beamercolorbox}[sep=12pt,center]{part title}
    \usebeamerfont{section title}\insertsection\par
  \end{beamercolorbox}
}
\setbeamertemplate{subsection page}{
  \centering
  \begin{beamercolorbox}[sep=8pt,center]{part title}
    \usebeamerfont{subsection title}\insertsubsection\par
  \end{beamercolorbox}
}
\AtBeginPart{
  \frame{\partpage}
}
\AtBeginSection{
  \ifbibliography
  \else
    \frame{\sectionpage}
  \fi
}
\AtBeginSubsection{
  \frame{\subsectionpage}
}

\usepackage{amsmath,amssymb}
\usepackage{lmodern}
\usepackage{iftex}
\ifPDFTeX
  \usepackage[T1]{fontenc}
  \usepackage[utf8]{inputenc}
  \usepackage{textcomp} % provide euro and other symbols
\else % if luatex or xetex
  \usepackage{unicode-math}
  \defaultfontfeatures{Scale=MatchLowercase}
  \defaultfontfeatures[\rmfamily]{Ligatures=TeX,Scale=1}
  \setmainfont[]{D-DIN}
  \setsansfont[]{Latin Modern Sans}
  \setmathfont[]{Latin Modern Math}
\fi
\usecolortheme{Flip}
\usefonttheme{serif} % use mainfont rather than sansfont for slide text
\useinnertheme{Flip}
\useoutertheme{Flip}
% Use upquote if available, for straight quotes in verbatim environments
\IfFileExists{upquote.sty}{\usepackage{upquote}}{}
\IfFileExists{microtype.sty}{% use microtype if available
  \usepackage[]{microtype}
  \UseMicrotypeSet[protrusion]{basicmath} % disable protrusion for tt fonts
}{}
\makeatletter
\@ifundefined{KOMAClassName}{% if non-KOMA class
  \IfFileExists{parskip.sty}{%
    \usepackage{parskip}
  }{% else
    \setlength{\parindent}{0pt}
    \setlength{\parskip}{6pt plus 2pt minus 1pt}}
}{% if KOMA class
  \KOMAoptions{parskip=half}}
\makeatother
\usepackage{xcolor}
\newif\ifbibliography
\setlength{\emergencystretch}{3em} % prevent overfull lines
\setcounter{secnumdepth}{-\maxdimen} % remove section numbering

\usepackage{color}
\usepackage{fancyvrb}
\newcommand{\VerbBar}{|}
\newcommand{\VERB}{\Verb[commandchars=\\\{\}]}
\DefineVerbatimEnvironment{Highlighting}{Verbatim}{commandchars=\\\{\}}
% Add ',fontsize=\small' for more characters per line
\usepackage{framed}
\definecolor{shadecolor}{RGB}{241,243,245}
\newenvironment{Shaded}{\begin{snugshade}}{\end{snugshade}}
\newcommand{\AlertTok}[1]{\textcolor[rgb]{0.68,0.00,0.00}{#1}}
\newcommand{\AnnotationTok}[1]{\textcolor[rgb]{0.37,0.37,0.37}{#1}}
\newcommand{\AttributeTok}[1]{\textcolor[rgb]{0.40,0.45,0.13}{#1}}
\newcommand{\BaseNTok}[1]{\textcolor[rgb]{0.68,0.00,0.00}{#1}}
\newcommand{\BuiltInTok}[1]{\textcolor[rgb]{0.00,0.23,0.31}{#1}}
\newcommand{\CharTok}[1]{\textcolor[rgb]{0.13,0.47,0.30}{#1}}
\newcommand{\CommentTok}[1]{\textcolor[rgb]{0.37,0.37,0.37}{#1}}
\newcommand{\CommentVarTok}[1]{\textcolor[rgb]{0.37,0.37,0.37}{\textit{#1}}}
\newcommand{\ConstantTok}[1]{\textcolor[rgb]{0.56,0.35,0.01}{#1}}
\newcommand{\ControlFlowTok}[1]{\textcolor[rgb]{0.00,0.23,0.31}{#1}}
\newcommand{\DataTypeTok}[1]{\textcolor[rgb]{0.68,0.00,0.00}{#1}}
\newcommand{\DecValTok}[1]{\textcolor[rgb]{0.68,0.00,0.00}{#1}}
\newcommand{\DocumentationTok}[1]{\textcolor[rgb]{0.37,0.37,0.37}{\textit{#1}}}
\newcommand{\ErrorTok}[1]{\textcolor[rgb]{0.68,0.00,0.00}{#1}}
\newcommand{\ExtensionTok}[1]{\textcolor[rgb]{0.00,0.23,0.31}{#1}}
\newcommand{\FloatTok}[1]{\textcolor[rgb]{0.68,0.00,0.00}{#1}}
\newcommand{\FunctionTok}[1]{\textcolor[rgb]{0.28,0.35,0.67}{#1}}
\newcommand{\ImportTok}[1]{\textcolor[rgb]{0.00,0.46,0.62}{#1}}
\newcommand{\InformationTok}[1]{\textcolor[rgb]{0.37,0.37,0.37}{#1}}
\newcommand{\KeywordTok}[1]{\textcolor[rgb]{0.00,0.23,0.31}{#1}}
\newcommand{\NormalTok}[1]{\textcolor[rgb]{0.00,0.23,0.31}{#1}}
\newcommand{\OperatorTok}[1]{\textcolor[rgb]{0.37,0.37,0.37}{#1}}
\newcommand{\OtherTok}[1]{\textcolor[rgb]{0.00,0.23,0.31}{#1}}
\newcommand{\PreprocessorTok}[1]{\textcolor[rgb]{0.68,0.00,0.00}{#1}}
\newcommand{\RegionMarkerTok}[1]{\textcolor[rgb]{0.00,0.23,0.31}{#1}}
\newcommand{\SpecialCharTok}[1]{\textcolor[rgb]{0.37,0.37,0.37}{#1}}
\newcommand{\SpecialStringTok}[1]{\textcolor[rgb]{0.13,0.47,0.30}{#1}}
\newcommand{\StringTok}[1]{\textcolor[rgb]{0.13,0.47,0.30}{#1}}
\newcommand{\VariableTok}[1]{\textcolor[rgb]{0.07,0.07,0.07}{#1}}
\newcommand{\VerbatimStringTok}[1]{\textcolor[rgb]{0.13,0.47,0.30}{#1}}
\newcommand{\WarningTok}[1]{\textcolor[rgb]{0.37,0.37,0.37}{\textit{#1}}}

\providecommand{\tightlist}{%
  \setlength{\itemsep}{0pt}\setlength{\parskip}{0pt}}\usepackage{longtable,booktabs,array}
\usepackage{calc} % for calculating minipage widths
\usepackage{caption}
% Make caption package work with longtable
\makeatletter
\def\fnum@table{\tablename~\thetable}
\makeatother
\usepackage{graphicx}
\makeatletter
\def\maxwidth{\ifdim\Gin@nat@width>\linewidth\linewidth\else\Gin@nat@width\fi}
\def\maxheight{\ifdim\Gin@nat@height>\textheight\textheight\else\Gin@nat@height\fi}
\makeatother
% Scale images if necessary, so that they will not overflow the page
% margins by default, and it is still possible to overwrite the defaults
% using explicit options in \includegraphics[width, height, ...]{}
\setkeys{Gin}{width=\maxwidth,height=\maxheight,keepaspectratio}
% Set default figure placement to htbp
\makeatletter
\def\fps@figure{htbp}
\makeatother

\usepackage{booktabs}
\usepackage{longtable}
\usepackage{array}
\usepackage{multirow}
\usepackage{wrapfig}
\usepackage{float}
\usepackage{colortbl}
\usepackage{pdflscape}
\usepackage{tabu}
\usepackage{threeparttable}
\usepackage{threeparttablex}
\usepackage[normalem]{ulem}
\usepackage{makecell}
\usepackage{xcolor}
\usepackage{tabu}
\usepackage{mathtools}
\usepackage{mathrsfs}
\makeatletter
\makeatother
\makeatletter
\makeatother
\makeatletter
\@ifpackageloaded{caption}{}{\usepackage{caption}}
\AtBeginDocument{%
\ifdefined\contentsname
  \renewcommand*\contentsname{Table of contents}
\else
  \newcommand\contentsname{Table of contents}
\fi
\ifdefined\listfigurename
  \renewcommand*\listfigurename{List of Figures}
\else
  \newcommand\listfigurename{List of Figures}
\fi
\ifdefined\listtablename
  \renewcommand*\listtablename{List of Tables}
\else
  \newcommand\listtablename{List of Tables}
\fi
\ifdefined\figurename
  \renewcommand*\figurename{Figure}
\else
  \newcommand\figurename{Figure}
\fi
\ifdefined\tablename
  \renewcommand*\tablename{Table}
\else
  \newcommand\tablename{Table}
\fi
}
\@ifpackageloaded{float}{}{\usepackage{float}}
\floatstyle{ruled}
\@ifundefined{c@chapter}{\newfloat{codelisting}{h}{lop}}{\newfloat{codelisting}{h}{lop}[chapter]}
\floatname{codelisting}{Listing}
\newcommand*\listoflistings{\listof{codelisting}{List of Listings}}
\makeatother
\makeatletter
\@ifpackageloaded{caption}{}{\usepackage{caption}}
\@ifpackageloaded{subcaption}{}{\usepackage{subcaption}}
\makeatother
\makeatletter
\@ifpackageloaded{tcolorbox}{}{\usepackage[many]{tcolorbox}}
\makeatother
\makeatletter
\@ifundefined{shadecolor}{\definecolor{shadecolor}{rgb}{.97, .97, .97}}
\makeatother
\makeatletter
\makeatother
\ifLuaTeX
  \usepackage{selnolig}  % disable illegal ligatures
\fi
\IfFileExists{bookmark.sty}{\usepackage{bookmark}}{\usepackage{hyperref}}
\IfFileExists{xurl.sty}{\usepackage{xurl}}{} % add URL line breaks if available
\urlstyle{same} % disable monospaced font for URLs
\hypersetup{
  pdftitle={Analyse de regroupements},
  pdfauthor={Léo Belzile},
  hidelinks,
  pdfcreator={LaTeX via pandoc}}

\title{Analyse de regroupements}
\subtitle{Analyse multidimensionnelle appliquée}
\author{Léo Belzile}
\date{automne 2022}
\institute{HEC Montréal}

\begin{document}
\frame{\titlepage}
\ifdefined\Shaded\renewenvironment{Shaded}{\begin{tcolorbox}[boxrule=0pt, interior hidden, borderline west={3pt}{0pt}{shadecolor}, enhanced, breakable, frame hidden, sharp corners]}{\end{tcolorbox}}\fi

\begin{frame}{Algorithmes pour l'analyse de regroupements}
\protect\hypertarget{algorithmes-pour-lanalyse-de-regroupements}{}
L'analyse de regroupements cherche à créer une division de \(n\)
observations de \(p\) variables en regroupements.

\begin{enumerate}
\tightlist
\item
  méthodes basées sur la connectivité (regroupements hiérarchiques,
  AGNES et DIANA)
\item
  méthodes basées sur les centroïdes et les médoïdes (\(k\)-moyennes,
  \(k\)-médoides PAM, CLARA)
\item
  mélanges de modèles (mélanges Gaussiens, etc.)
\item
  méthodes basées sur la densité (DBScan)
\item
  méthodes spectrales
\end{enumerate}
\end{frame}

\begin{frame}[fragile]{\(K\)-médianes}
\protect\hypertarget{k-muxe9dianes}{}
Illustration de la répartition/interprétation avec \(K=5\) groupes

\begin{Shaded}
\begin{Highlighting}[numbers=left,,]
\FunctionTok{set.seed}\NormalTok{(}\DecValTok{60602}\NormalTok{)}
\NormalTok{kmed5 }\OtherTok{\textless{}{-}}\NormalTok{ flexclust}\SpecialCharTok{::}\FunctionTok{kcca}\NormalTok{(}
  \AttributeTok{x =}\NormalTok{ donsmult\_std,}
  \AttributeTok{k =} \DecValTok{5}\NormalTok{,}
  \AttributeTok{family =}\NormalTok{ flexclust}\SpecialCharTok{::}\FunctionTok{kccaFamily}\NormalTok{(}\StringTok{"kmedians"}\NormalTok{),}
  \AttributeTok{control =} \FunctionTok{list}\NormalTok{(}\AttributeTok{initcent =} \StringTok{"kmeanspp"}\NormalTok{))}
\end{Highlighting}
\end{Shaded}
\end{frame}

\begin{frame}{Différences de segmentation}
\protect\hypertarget{diffuxe9rences-de-segmentation}{}
\begin{figure}

{\centering \includegraphics[width=0.8\textwidth,height=\textheight]{MATH60602-diapos12_files/figure-beamer/fig-acpkmoy5-1.pdf}

}

\caption{\label{fig-acpkmoy5}Nuage de points des deux premières
composantes principales des observations de dons multiples avec les
étiquettes des regroupements obtenus selon la méthodes des
\(K\)-moyennes et \(K\)-médianes avec \(K=5\) regroupements.}

\end{figure}
\end{frame}

\begin{frame}{Commentaire sur \(K\)-médianes}
\protect\hypertarget{commentaire-sur-k-muxe9dianes}{}
Avec les \(K\)-médianes, les personnes qui ont fait des dons plus élevés
sont fusionnés avec d'autres personnes qui ont fait des dons moins
élevés et les groupes sont davantage de taille comparable.

Selon l'objectif des regroupements, cela peut être avantageux, mais
cibler les donateurs les plus généreux semble plus logique dans le
contexte.
\end{frame}

\begin{frame}{\(K\)-médoïdes.}
\protect\hypertarget{k-muxe9douxefdes.}{}
Dans les \(K\)-médoïdes, on choisit une observation comme prototype.

Puisque qu'on considère chaque observation comme candidat à devenir un
médoïde à chaque étape, le coût de calcul est prohibitif en grande
dimension.
\end{frame}

\begin{frame}{Algorithme de partition autour des médoïdes (PAM)}
\protect\hypertarget{algorithme-de-partition-autour-des-muxe9douxefdes-pam}{}
\begin{enumerate}
\tightlist
\item
  Initialisation: sélectionner \(K\) des \(n\) observations comme
  médoïdes initiaux.
\item
  Assigner chaque observation au médoïde le plus près.
\item
  Calculer la dissimilarité totale entre chaque médoïde et les
  observations de son groupe.
\item
  Pour chaque médoïde \((k=1, \ldots, K\)):

  \begin{itemize}
  \tightlist
  \item
    considérer tous les \(n-K\) observations à tour de rôle et permuter
    le médoïde avec l'observation.\\
  \item
    calculer la distance totale et sélectionner l'observation qui
    diminue le plus la distance totale.
  \end{itemize}
\item
  Répéter les étapes 2 à 4 jusqu'à ce que les médoïdes ne changent plus.
\end{enumerate}
\end{frame}

\begin{frame}{Algorithme CLARA (1/2)}
\protect\hypertarget{algorithme-clara-12}{}
L'algorithme CLARA, décrit dans Kaufman \& Rousseeuw (1990), réduit le
coût de calcul et de stockage en

\begin{itemize}
\tightlist
\item
  divisant l'échantillon en \(S\) sous-échantillons de taille
  approximativement égale (par défaut \(S=5\))
\item
  et en utilisant l'algorithme PAM sur chacun.
\end{itemize}

Une fois les médoïdes obtenus, le reste de toutes les observations de
l'échantillon sont assignées au regroupement du médoïde le plus près.
\end{frame}

\begin{frame}{Algorithme CLARA (2/2)}
\protect\hypertarget{algorithme-clara-22}{}
La qualité de la segmentation pour chacune des segmentations est
calculée en obtenant la distance moyenne entre les médoïdes et les
observations.

On retourne la meilleure (celle qui a la plus petite distance moyenne).
\end{frame}

\begin{frame}[fragile]{PAM et CLARA dans \textbf{R}}
\protect\hypertarget{pam-et-clara-dans-r}{}
Disponible depuis le paquet \texttt{cluster}.

\begin{Shaded}
\begin{Highlighting}[numbers=left,,]
\FunctionTok{set.seed}\NormalTok{(}\DecValTok{60602}\NormalTok{)}
\NormalTok{kmedoide5 }\OtherTok{\textless{}{-}}\NormalTok{ cluster}\SpecialCharTok{::}\FunctionTok{clara}\NormalTok{(}
   \AttributeTok{x =}\NormalTok{ donsmult\_std,}
   \AttributeTok{k =}\NormalTok{ 5L, }\CommentTok{\# nombre de groupes}
   \AttributeTok{sampsize =} \DecValTok{500}\NormalTok{, }\CommentTok{\#taille échantillon pour PAM}
   \AttributeTok{metric =} \StringTok{"euclidean"}\NormalTok{, }\CommentTok{\# distance l2}
   \CommentTok{\#cluster.only = TRUE, \# ne conserver que étiquettes}
   \AttributeTok{rngR =} \ConstantTok{TRUE}\NormalTok{, }\CommentTok{\# germe aléatoire depuis R}
   \AttributeTok{pamLike =} \ConstantTok{TRUE}\NormalTok{, }\CommentTok{\# même algorithme que PAM}
   \AttributeTok{samples =} \DecValTok{10}\NormalTok{) }\CommentTok{\#nombre de répétitions aléatoires}
\end{Highlighting}
\end{Shaded}
\end{frame}

\begin{frame}{Avantages et inconvénients des \(K\)-médoïdes}
\protect\hypertarget{avantages-et-inconvuxe9nients-des-k-muxe9douxefdes}{}
\begin{itemize}
\tightlist
\item
  (+) les prototypes sont des observations de l'échantillon.
\item
  (+) la fonction objective est moins impactée par les extrêmes.
\item
  (-) le coût de calcul est prohibitif avec des mégadonnées (problème
  combinatoire). PAM fonctionne avec maximum 1000 observations.
\end{itemize}
\end{frame}

\begin{frame}{Valeurs initiales et paramètres}
\protect\hypertarget{valeurs-initiales-et-paramuxe8tres}{}
Même hyperparamètres que \(K\)-moyennes (dissemblance, nombre de
regroupements, initialisation et séparation).

Comme les \(K\)-moyennes, on fera plusieurs essais pour trouver de
bonnes valeurs de départ. On peut tracer le profil des silhouettes
(Figure~\ref{fig-clarasilhouette})

\begin{figure}

{\centering \includegraphics[width=0.8\textwidth,height=\textheight]{MATH60602-diapos12_files/figure-beamer/fig-clarasilhouette-1.pdf}

}

\caption{\label{fig-clarasilhouette}Silhouettes pour les données de dons
multiples avec l'algorithme CLARA pour \(K=5\) regroupements.}

\end{figure}
\end{frame}

\begin{frame}[fragile]{Prototypes}
\protect\hypertarget{prototypes}{}
Puisque les prototypes (médoïdes) sont des observations, on peut
simplement extraire leur identifiant

\begin{Shaded}
\begin{Highlighting}[numbers=left,,]
\NormalTok{medoides\_orig }\OtherTok{\textless{}{-}}\NormalTok{ donsmult[kmedoide[[}\DecValTok{4}\NormalTok{]]}\SpecialCharTok{$}\NormalTok{i.med,]}
\NormalTok{medoides\_orig}
\CommentTok{\# Taille des regroupements}
\NormalTok{kmedoide[[}\DecValTok{4}\NormalTok{]]}\SpecialCharTok{$}\NormalTok{clusinfo}
\end{Highlighting}
\end{Shaded}
\end{frame}

\begin{frame}{Mesures de similarité}
\protect\hypertarget{mesures-de-similarituxe9}{}
Certains algorithmes utilisent directement une matrice de similarité
\(\mathbf{S}\) qui encode plutôt l'information à propos des points
avoisinants.

Plus les observations sont similaires, plus elles sont proches.

Les mesures dissemblance peuvent être convertie en mesure de similarité.

En haute dimension, il est intéressant d'obtenir une matrice de
similarité creuse (avec beaucoup de zéros).

\begin{itemize}
\tightlist
\item
  Méthodes de noyau à support compact
\item
  voisinage \(\epsilon\): toute paire d'observation à distance au plus
  \(\epsilon\) a une similarité de \(s=1\) et \(s=0\) sinon.
\item
  \(k\) plus proches voisins: similarité de \(S_{ij}=1\) si
  l'observation \(\mathbf{X}_j\) est un des \(k\) plus proches voisins
  d'observation \(\mathbf{X}_i\) (ou vice-versa)
\end{itemize}
\end{frame}



\end{document}
