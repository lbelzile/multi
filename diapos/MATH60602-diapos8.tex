% Options for packages loaded elsewhere
\PassOptionsToPackage{unicode}{hyperref}
\PassOptionsToPackage{hyphens}{url}
%
\documentclass[
  ignorenonframetext,
]{beamer}
\usepackage{pgfpages}
\setbeamertemplate{caption}[numbered]
\setbeamertemplate{caption label separator}{: }
\setbeamercolor{caption name}{fg=normal text.fg}
\beamertemplatenavigationsymbolsempty
% Prevent slide breaks in the middle of a paragraph
\widowpenalties 1 10000
\raggedbottom
\setbeamertemplate{part page}{
  \centering
  \begin{beamercolorbox}[sep=16pt,center]{part title}
    \usebeamerfont{part title}\insertpart\par
  \end{beamercolorbox}
}
\setbeamertemplate{section page}{
  \centering
  \begin{beamercolorbox}[sep=12pt,center]{part title}
    \usebeamerfont{section title}\insertsection\par
  \end{beamercolorbox}
}
\setbeamertemplate{subsection page}{
  \centering
  \begin{beamercolorbox}[sep=8pt,center]{part title}
    \usebeamerfont{subsection title}\insertsubsection\par
  \end{beamercolorbox}
}
\AtBeginPart{
  \frame{\partpage}
}
\AtBeginSection{
  \ifbibliography
  \else
    \frame{\sectionpage}
  \fi
}
\AtBeginSubsection{
  \frame{\subsectionpage}
}

\usepackage{amsmath,amssymb}
\usepackage{lmodern}
\usepackage{iftex}
\ifPDFTeX
  \usepackage[T1]{fontenc}
  \usepackage[utf8]{inputenc}
  \usepackage{textcomp} % provide euro and other symbols
\else % if luatex or xetex
  \usepackage{unicode-math}
  \defaultfontfeatures{Scale=MatchLowercase}
  \defaultfontfeatures[\rmfamily]{Ligatures=TeX,Scale=1}
  \setmainfont[]{D-DIN}
  \setsansfont[]{Latin Modern Sans}
  \setmathfont[]{Latin Modern Math}
\fi
\usecolortheme{Flip}
\usefonttheme{serif} % use mainfont rather than sansfont for slide text
\useinnertheme{Flip}
\useoutertheme{Flip}
% Use upquote if available, for straight quotes in verbatim environments
\IfFileExists{upquote.sty}{\usepackage{upquote}}{}
\IfFileExists{microtype.sty}{% use microtype if available
  \usepackage[]{microtype}
  \UseMicrotypeSet[protrusion]{basicmath} % disable protrusion for tt fonts
}{}
\makeatletter
\@ifundefined{KOMAClassName}{% if non-KOMA class
  \IfFileExists{parskip.sty}{%
    \usepackage{parskip}
  }{% else
    \setlength{\parindent}{0pt}
    \setlength{\parskip}{6pt plus 2pt minus 1pt}}
}{% if KOMA class
  \KOMAoptions{parskip=half}}
\makeatother
\usepackage{xcolor}
\newif\ifbibliography
\setlength{\emergencystretch}{3em} % prevent overfull lines
\setcounter{secnumdepth}{-\maxdimen} % remove section numbering


\providecommand{\tightlist}{%
  \setlength{\itemsep}{0pt}\setlength{\parskip}{0pt}}\usepackage{longtable,booktabs,array}
\usepackage{calc} % for calculating minipage widths
\usepackage{caption}
% Make caption package work with longtable
\makeatletter
\def\fnum@table{\tablename~\thetable}
\makeatother
\usepackage{graphicx}
\makeatletter
\def\maxwidth{\ifdim\Gin@nat@width>\linewidth\linewidth\else\Gin@nat@width\fi}
\def\maxheight{\ifdim\Gin@nat@height>\textheight\textheight\else\Gin@nat@height\fi}
\makeatother
% Scale images if necessary, so that they will not overflow the page
% margins by default, and it is still possible to overwrite the defaults
% using explicit options in \includegraphics[width, height, ...]{}
\setkeys{Gin}{width=\maxwidth,height=\maxheight,keepaspectratio}
% Set default figure placement to htbp
\makeatletter
\def\fps@figure{htbp}
\makeatother

\usepackage{tabu}
\usepackage{mathtools}
\usepackage{mathrsfs}
\makeatletter
\makeatother
\makeatletter
\makeatother
\makeatletter
\@ifpackageloaded{caption}{}{\usepackage{caption}}
\AtBeginDocument{%
\ifdefined\contentsname
  \renewcommand*\contentsname{Table of contents}
\else
  \newcommand\contentsname{Table of contents}
\fi
\ifdefined\listfigurename
  \renewcommand*\listfigurename{List of Figures}
\else
  \newcommand\listfigurename{List of Figures}
\fi
\ifdefined\listtablename
  \renewcommand*\listtablename{List of Tables}
\else
  \newcommand\listtablename{List of Tables}
\fi
\ifdefined\figurename
  \renewcommand*\figurename{Figure}
\else
  \newcommand\figurename{Figure}
\fi
\ifdefined\tablename
  \renewcommand*\tablename{Table}
\else
  \newcommand\tablename{Table}
\fi
}
\@ifpackageloaded{float}{}{\usepackage{float}}
\floatstyle{ruled}
\@ifundefined{c@chapter}{\newfloat{codelisting}{h}{lop}}{\newfloat{codelisting}{h}{lop}[chapter]}
\floatname{codelisting}{Listing}
\newcommand*\listoflistings{\listof{codelisting}{List of Listings}}
\makeatother
\makeatletter
\@ifpackageloaded{caption}{}{\usepackage{caption}}
\@ifpackageloaded{subcaption}{}{\usepackage{subcaption}}
\makeatother
\makeatletter
\@ifpackageloaded{tcolorbox}{}{\usepackage[many]{tcolorbox}}
\makeatother
\makeatletter
\@ifundefined{shadecolor}{\definecolor{shadecolor}{rgb}{.97, .97, .97}}
\makeatother
\makeatletter
\makeatother
\ifLuaTeX
  \usepackage{selnolig}  % disable illegal ligatures
\fi
\IfFileExists{bookmark.sty}{\usepackage{bookmark}}{\usepackage{hyperref}}
\IfFileExists{xurl.sty}{\usepackage{xurl}}{} % add URL line breaks if available
\urlstyle{same} % disable monospaced font for URLs
\hypersetup{
  pdftitle={Régression logistique: prédictions et données multinomiales},
  pdfauthor={Léo Belzile},
  hidelinks,
  pdfcreator={LaTeX via pandoc}}

\title{Régression logistique: prédictions et données multinomiales}
\subtitle{Analyse multidimensionnelle appliquée}
\author{Léo Belzile}
\date{automne 2022}
\institute{HEC Montréal}

\begin{document}
\frame{\titlepage}
\ifdefined\Shaded\renewenvironment{Shaded}{\begin{tcolorbox}[sharp corners, breakable, frame hidden, interior hidden, enhanced, borderline west={3pt}{0pt}{shadecolor}, boxrule=0pt]}{\end{tcolorbox}}\fi

\begin{frame}{Sélection de variables en régression logistique}
\protect\hypertarget{suxe9lection-de-variables-en-ruxe9gression-logistique}{}
Même principes que précédemment, mais les modèles de régression
logistique sont plus coûteux à estimer.

On peut utiliser les critères d'information puisque le modèle est ajusté
par maximum de vraisemblance.
\end{frame}

\begin{frame}[fragile]{Fonctions \textbf{R} pour la sélection de
modèles}
\protect\hypertarget{fonctions-r-pour-la-suxe9lection-de-moduxe8les}{}
\begin{itemize}
\tightlist
\item
  \texttt{glmbb::glmbb} permet une recherche exhaustive de tous les
  sous-modèles à au plus une certaine distance (\texttt{cutoff}) du
  modèle avec le plus petit critère d'information (\texttt{criterion}).
\item
  \texttt{step} permet de faire une recherche séquentielle avec un
  critère d'information.
\item
  \texttt{glmulti::glmulti} permet une recherche exhaustive
  (\texttt{method\ =\ "h"}) ou par le biais d'un algorithme génétique
  (\texttt{method\ =\ "g"}).
\item
  \texttt{glmnet::glmnet} permet d'ajuster le modèle avec pénalité
  LASSO.
\end{itemize}

Voir le
\href{https://lbelzile.github.io/math60602/05-reglogistique.html\#s\%C3\%A9lection-de-variables-en-r\%C3\%A9gression-logistique}{code
en ligne}.
\end{frame}

\begin{frame}{Objectif du ciblage marketing}
\protect\hypertarget{objectif-du-ciblage-marketing}{}
Déterminer si le revenu prévu justifie l'envoi du catalogue

\[\mathsf{E}(\textsf{ymontant}_i) = \mathsf{E}(\textsf{ymontant}_i \mid \textsf{yachat}_i = 1)\Pr(\textsf{yachat}_i = 1).\]

On peut combiner un modèle de régression logistique avec la régression
linéaire (ajustés simultanément avec un modèle Heckit).

Ou simplement ignorer le montant d'achat et envoyer un catalogue si la
probabilité d'achat excède notre point de coupure optimal.
\end{frame}

\begin{frame}{Stratégie de référence}
\protect\hypertarget{stratuxe9gie-de-ruxe9fuxe9rence}{}
\begin{itemize}
\tightlist
\item
  Parmi les 100K clients, 23 179 auraient acheté si on leur avait envoyé
  le catalogue
\item
  Ces clients auraient généré des revenus de 1 601 212\$.
\item
  Si on enlève le coût des envois (100 000 \(\times\) 10\$), la
  stratégie de référence permet un revenu net de 601 212\$.
\end{itemize}
\end{frame}

\begin{frame}{Stratégie d'ajustement}
\protect\hypertarget{stratuxe9gie-dajustement}{}
En résumé, la procédure numérique à réaliser est la suivante:

\begin{itemize}
\tightlist
\item
  Choisir les variables à essayer (termes quadratiques, interactions,
  etc.)
\item
  Choisir l'algorithme ou la méthode de sélection du modèle.
\item
  Construire un catalogue de modèles: pour chacun, calculer les
  prédictions par validation croisée.
\item
  Calculer le point de coupure optimal pour chaque modèle et le gain
  associé.
\item
  Sélectionner le modèle qui \textbf{maximise le gain}.
\end{itemize}
\end{frame}

\begin{frame}{Prédiction et envoi}
\protect\hypertarget{pruxe9diction-et-envoi}{}
\begin{itemize}
\tightlist
\item
  Prédire les 100 000 observations de l'échantillon test.
\item
  Envoyer un catalogue si la probabilité d'achat excède le point de
  coupure.
\item
  Calculer le revenu résultant:

  \begin{itemize}
  \tightlist
  \item
    zéro si on n'envoie pas de catalogue
  \item
    \(-10\) si la personne n'achète pas
  \item
    \(-10\) plus l'achat si la personne achète.
  \end{itemize}
\end{itemize}

\textbf{En pratique}, on ne pourrait pas \emph{a priori} connaître le
revenu résultant de cette stratégie.
\end{frame}

\begin{frame}{Conclusion}
\protect\hypertarget{conclusion}{}
Si on avait fait une bête recherche séquentielle et qu'on avait pris le
modèle avec le plus petit BIC (8 variables explicatives), on aurait
dégagé des revenus de 978 226\$.

C'est une énorme amélioration, de plus de 56\%, par rapport à la
stratégie de référence.
\end{frame}

\begin{frame}{Récapitulatif}
\protect\hypertarget{ruxe9capitulatif}{}
\begin{itemize}
\tightlist
\item
  Les principes de sélection de variable couverts précédemment
  s'appliquent toujours (recherche exhaustive, séquentielle et LASSO).
\item
  On peut calculer les critères d'information puisque le modèle est
  ajusté par maximum de vraisemblance.
\item
  Attention au surajustement! Suspect si les probabilités estimées sont
  près de 0/1 (vérifier la calibration).
\item
  Deux étapes: sélectionner le modèle (variables) et le point de
  coupure.
\item
  D'autres modèles que la régression logistique (arbres de régression,
  etc.) sont envisageables pour la classification.
\end{itemize}
\end{frame}



\end{document}
